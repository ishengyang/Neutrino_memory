\documentclass[aps,showpacs,onecolumn,floats,prd,superscriptaddress,nofootinbib]{revtex4-1} 
\usepackage{graphicx,amsmath,amssymb,amstext}
\usepackage{amssymb,amsbsy,amsfonts,amsthm,color}
\usepackage{epsfig}
%\usepackage{showkeys}
\usepackage{graphicx}
\usepackage{subfigure}
\graphicspath{{Figures/}}

\begin{document}

\title{Longitudinal gravitational memory}

\author{Darsh Kodwani}
\email{dkodwani@physics.utoronto.ca}
\affiliation{Canadian Institute of Theoretical Astrophysics, 60 St George St, Toronto, ON M5S 3H8, Canada.}
\affiliation{University of Toronto, Department of Physics, 60 St George St, Toronto, ON M5S 3H8, Canada.}

\author{Ue-Li Pen}
\email{pen@cita.utoronto.ca}
\affiliation{Canadian Institute of Theoretical Astrophysics, 60 St George St, Toronto, ON M5S 3H8, Canada.}
\affiliation{Canadian Institute for Advanced Research, CIFAR program in Gravitation and Cosmology.}
\affiliation{Dunlap Institute for Astronomy \& Astrophysics, University of Toronto, AB 120-50 St. George Street, Toronto, ON M5S 3H4, Canada.}
\affiliation{Perimeter Institute of Theoretical Physics, 31 Caroline Street North, Waterloo, ON N2L 2Y5, Canada.}

\author{I-Sheng Yang}
\email{isheng.yang@gmail.com}
\affiliation{Canadian Institute of Theoretical Astrophysics, 60 St George St, Toronto, ON M5S 3H8, Canada.}
\affiliation{Perimeter Institute of Theoretical Physics, 31 Caroline Street North, Waterloo, ON N2L 2Y5, Canada.}

\begin{abstract}

We calculate the gravitational memory effect when a spherically symmetric shell of energy passes through a spacetime region. In particular, this effect includes a longitudinal component, such that two geodesics can pick up a relative velocity proportional to their separation. Such a measurement will allow us to obtain the total energy released by a supernova explosion in the form of neutrinos. We study the possibility to measure such an effect by space-based interferometers such as LISA and BBO, and also by astrophysical interferometers such as pulsar scintillometry.

\end{abstract}

\maketitle

\section{Introduction and Summary}

The recent detection of gravitational waves \cite{GW1509} has proved that gravitational waves leave an oscillating pattern in the amplitude of waveforms measured at detectors such as LIGO. It is also known that this is not the only effect that is potentially detectable. Strong gravitational waves imply a large flow of energy. Just like any other flow of energy, it leads to a gravitational memory effect \cite{Christodoulou_effect,GW_memory}. 

\begin{figure}[b]
\begin{center}
\includegraphics[scale = 0.27]{intro.pdf}
\caption{Schematic of the effect being considered by a neutrino shell passing through the interferometer. The points $A,B,C$ represent ends of the interferometer of arm length $d$. The three points $A,B$ and $C$ will pick up velocities $v_A, v_B$ and $v_C$ respectively after the shell crosses them. They are all different since they cross the shell at different locations.}
\label{fig:1}
\end{center}
\end{figure}

The memory effect discussed in \cite{Christodoulou_effect,GW_memory} causes permanent relative displacements between geodesics. It contains only transverse-traceless components and can leave an imprint in an interferometer. In this paper, we will introduce another memory effect that is different in two ways:
\begin{itemize}
\item It has a longitudinal component. The transverse-traceless limitation only applies to freely-propagating changes of the metric (i.e gravitational waves). While coupled to matter, which often have longitudinal (density) waves, it is natural to have an accompanying longitudinal change in the metric.
\item In addition to displacements of geodesics, it also causes permanent changes in relative velocities between geodesics, with magnitudes proportional to their separations.
\end{itemize}

In terms of the dynamics, a change in velocity is higher order than a change in distance measured in gravitational waves. This however does not mean that our effect is harder to measure. The conventional gravitational memory effect needs a physical event that significantly breaks spherical symmetry to be detectable whereas our effect does not and therefore it can occur more generally. In addition, a change in velocity implies a distance change that grows in time even after the initial effect. That is an advantage for some detection methods.  

We start by analyzing an astrophysical scenario inspired by previous work in \cite{PhysRevD.93.103006} . We calculate what happens to geodesics that are parallel when an energy shell crosses them. The particular set-up we consider is a neutrino shell emitted by supernova. The shell crosses the path of light rays coming from a pulsar that is scintillating due to lensing in the interstellar medium. 
The SKA telescope is expected to detect thousands of pulsars, \cite{MSPpopulation}, and one of them might be close enough to a supernova explosion to provide us with a good estimation of the total energy in the neutrino shell of the explosion. Currently there is no other way to obtain such information. Thus in addition to direct neutrino detections such as in Super-Kamiokande \cite{SuperKSN}, this memory effect can provide a new handle on constraining the explosion mechanism. 

Following from this we consider what happens if one considers an interferometer such as LISA instead of the two geodesics of the pulsar photons. Our calculation will show that the ends of the interferometer will see a change in their geodesic distance: this is a memory effect. When the interferometer arm is oriented perpendicular to the line of sight of the supernova we will find a more interesting effect that grows linearly with time. 

%We will present a simple and natural occurrence of this effect. During a supernova explosion (SNe), most of the energy is released in a highly relativistic shell of neutrinos. As illustrated in Fig.\ref{fig:1}, when a neutrino shell passes through, the three free-falling points $A$, $B$ and $C$, will pick up different velocities due to the change of geometry. If $AB$ and $BC$ are two arms of an interferometer, we will see a time-dependent change in the phase of the photons in the interferometer after the shell passes through. Note that this is purely a geometric change which happens even without the three actual objects. We will demonstrate this by showing that it is possibility to detect the same effect using pulsar interferometry, in which two parallel light rays get different time-delays after being hit by a neutrino shell. 

The rest of the paper is organized as follows. In section \ref{sec-scint} we discuss how an astrophysical interferometer formed by pulsar scintillometry can measure the memory effect caused by a supernova. In section \ref{RelV} we derive the change in velocities caused by shell crossing the ends of an interferometer. This is done using the Israel Junction Conditions (IJC) \cite{Isr66} and by treating the neutrino shell as a co-dimension-one delta function. In the final section \ref{obs} we discuss potential observation of such an effect by experiments that are currently being planned such as LISA and BBO.

\section{Memory effect in Pulsar Scintillation}
\label{sec-scint}

\begin{figure}[h]
\begin{center}
\includegraphics[width=\textwidth,height=10cm]{Lens.pdf}
\caption{Geometry of the astrophysical interferometer formed by pulsar scintillometry. Due to scattering or lensing, the image we see is an interference pattern of two light rays represented by the blue lines. If the separation of the two light rays has a component along the longitudinal (radial) direction from the SN, the spacetime distortion of the neutrino shell will change the interference pattern we see. We draw the lens to be behind the SN, but it could have been in front of it and the effect is the same.}
\label{fig:4}
\end{center}
\end{figure}

It is known that the images of many astronomical bodies scintillate \cite{PulsarScint}. A general reason for scintillation is that due to scattering or lensing, we receive multiple light rays from the same objects. These light rays are very close to each other, so they cannot be individually resolved and have to interfere. The scintillation pattern we see is the time dependence of their interference. If we consider two light rays from a faraway pulsar which happen to pass by a SN progenitor, as illustrated in Fig.\ref{fig:4}, they can probe the spacetime distortion when it explodes. 

The scintillation/interference pattern is directly related to the path lengths of these light rays. The change in such path length during a SN explosion has been worked out in \cite{Olum:2013gza}
\begin{equation}
	\Delta t = 2\delta M \left[ \ln \left(1 + \frac{t^2}{b^2} \right) - \frac{t^2}{b^2 + t^2} \right].
\end{equation}
Here $b$ is the impact parameter as shown in Fig.\ref{fig:4}, the shortest distance between the light ray and the SN. $t$ is the proper time on earth, with $t=0$ the time we directly observe the SN explosion. $\delta M$ is the total energy of the neutrino shell, and $\Delta t$ is the resulting time shift. A photon which should have reached the earth at time $t$, will arrive earlier at $(t-\Delta t)$ instead.

When the separation between two light rays has a component in the radial direction from the SN, $\Delta b$, there will be a nonzero relative change between their path lengths.
\begin{equation}
	(\Delta t|_b - \Delta t|_{b+\Delta b}) \approx 
	\frac{\partial \Delta t}{\partial b} \Delta b 
	= - \frac{4\delta Mt^4}{b(b^2 + t^2)^2} \Delta b~.
	\label{eq-change}
\end{equation}
We can see that this effect grows from zero and approaches an asymptotic value,
\begin{equation}
	(\Delta t|_b - \Delta t|_{b+\Delta b}) 
	\longrightarrow \frac{4\delta M \Delta b}{b}~,	\label{pulsescint}
\end{equation}
at a characteristic time scale given by $b$.

We estimate $b$ by assuming that the next SN is somewhere near the galactic centre. A sample of $\sim 9000$ pulsars from the SKA catalog in \cite{MSPpopulation} shows that among those pulsars, the shortest $b$ is about $10\ ly \sim 10^{14} \ km$. $\Delta b$ is related to the scattering-broadening of images. We use the data from \cite{BowBel13} that was observed on a scattering screen near the galactic centre. Scaling the frequency to $1\ GHz$ which is usually a good window to observe pulsar signals. We found that such a scattering screen can produce images separated by $\Delta b\sim 1000A.U. \sim 10^{10} \ km$. We again use $\delta M \sim 1 \ km$, and combining all these numbers we get $(\delta M \Delta b / b) \sim 1 \ m$. This is comparable to the wavelength at $1 \ GHz$, thus making the change in interference pattern easy to detect. Therefore, if we can monitor the pulsar scintillation pattern over ten years after a SN explosion we should see an order one change in the scintillation pattern predicted by Eq.~(\ref{eq-change}). 


\section{Velocity change from junction conditions}
\label{RelV}

\begin{figure}[h]
\begin{center}
\includegraphics[width =\textwidth, height = 12cm]{calculation.pdf}
\caption{Spacetime diagram showing the paths of photons that are used in the interferometer to measure the change in the length of the interferometer arms. The orange lines represent the photon trajectories. The solid blue lines represent the trajectories of the two points $A$ and $B$ in figure \ref{fig:1}. The dark black line is the null trajectory of the neutrino shell. the proper lengths before(after) shell crossing are $\bar{l}_{AB}(l_{AB})$. Proper times before(after) shell crossing are $\bar{\tau}_{AB}(\tau_{AB})$. The coordinate time at which the shell crosses point $B$ is different in both metrics. $t_c$ in metric of mass $M-\delta M$ and $\bar{t}_c$ in the metric of mass $M$. }
\label{calculation}
\end{center}
\end{figure}
In this section we discuss how a similar effect can be also seen in traditional interferometers such as LIGO. We assume the geometry of the spacetime is governed by the SN progenitor star. Assuming the star is not rotating very rapidly and thus the surrounding spacetime is parametrized by the Schwarzschild metric. In the calculations we will need two Schwarzschild metrics. One with mass $M$ and the other with mass $M - \delta M$, where $\delta M$ will be the energy carried by the neutrinos after the SN explosion. The metric with mass $M$ will have a bar over its coordinates and the one with $M- \delta M$ will not. 
 
\begin{equation}
	d\bar{s}^2 = \bar{g}_{\mu \nu} d\bar{x}^\mu d\bar{x}^\nu = - \left( 1 - \frac{2M}{r} \right) d\bar{t}^2 + \left( 1 - \frac{2M}{r} \right)^{-1} d {r}^2 + r^2 d {\Omega}_2^2. \label{SCH}
\end{equation}

\begin{equation}
	ds^2 = g_{\mu \nu} dx^\mu dx^\nu = - \left( 1 - \frac{(M - \delta M)}{r} \right) dt^2 + \left( 1 - \frac{2(M - \delta M)}{r} \right)^{-1} dr^2 + r^2 d \Omega_2^2.
\end{equation}
We are working in units with $G = c =1$.  Notice that the time component of the metric is different in both geometries whereas the radial component is the same as it corresponds to the radius of a two sphere separating the two geometries.  Since we are describing the shell as a delta function travelling at roughly the speed of light it will follow a null geodesic. The null vector of the shell can be written in both metrics as follows\footnote{One could write this vector in different ways and still have it satisfy the null normalization condition, however, the fact the shell has some surface area with a \emph{fixed} radius, the two vectors must have the same radial component and this enforces the given form of the time component.}
\begin{equation}
	k^\mu = \left( g_{tt}^{-1}, 1, 0, 0 \right)	\label{sigma}
\end{equation}
and similarly for $\bar{k}^\mu$. Since we expect the interferometer points (denoted by the points $A,B,C$ in figure \ref{fig:1}) to be very far for the massive object we assume they have a negligible initial velocity.
\begin{equation}
	\bar{u}^\mu = \left( (-1/\bar{g}_{tt})^\frac{1}{2}, 0, 0, 0 \right).	\label{zeta}
\end{equation}
The $r$ will be different for points $A,B$ and $C$ as shown in figure \ref{fig:1}. After the shell has crossed the points will pick up a velocity $v$. We can calculate $v$ using the IJC, 
\begin{equation}
	 g_{\mu \nu} u^\mu k^\nu = \bar{g}_{\mu \nu} \bar{u}^\mu \bar{k}^\nu,	\label{0IJC}
\end{equation}

This expression comes from demanding the continuity of geodesics from one geometry to another and was used in \cite{PhysRevD.93.103006} to calculate a similar effect. 
Throughout this calculation we assume that the interferometer is very far from the SNe. Thus we assume there are three small quantities, $\frac{M}{r_0}, \frac{\delta M}{r_0}, \frac{d}{r_0} \ll1$. To simplify notation we will use the symbol $\mathcal{O}(r^{-1})$ to represent suppression by any one of the three small quantities. 
Substituting Eqs (\ref{sigma}, \ref{zeta}) into Eq (\ref{0IJC}) gives the final vector for the interferometer, to order $\mathcal{O}(r_{crossing}^{-2})$,
\begin{equation}
	u^\mu = \left( (-1/g_{tt})^\frac{1}{2} \sqrt{1-v^2g_{tt}^{-1}}, v , 0, 0 \right).	\label{5}
\end{equation}
The change in velocity due a shell crossing is $v = -\frac{\delta M}{r_{crossing}} \bar{g}_{tt}^{-\frac{1}{2}} $ . $r_{crossing}$ is a fixed distance at which the shell crosses a point. For $A$ it is $r_0$, for $B$ it is $r_0 + d$ and for $C$ it is $r_0 + \mathcal{O}(d/r_0)$. Note that this is the proper velocity $\frac{dr}{d\tau}$. We will need the coordinate velocity for the our calculations which is given by 

\begin{eqnarray}
	\frac{dr}{dt} &=& \frac{dr}{d \tau} \frac{d \tau}{dt} = v g_{tt}^{\frac{1}{2}} =  \frac{\delta M}{r_{crossing}} \bar{g}_{tt}^{-\frac{1}{2}} g_{tt}^\frac{1}{2} \nonumber \\
	&=& -\frac{\delta M}{r_{crossing}} \left( 1 + \frac{\delta M}{r_{crossing}} \right)
\end{eqnarray}

%
%\subsection{The relationship between proper time and distance}\label{conceptsec}
%
%The distance change is measured by an interferometer. The interferometer calculates the distance by sending photons to a target and measuring the time it takes for them to come back. To demonstrate this we consider the simple case shown in figure \ref{fig:3}. Two freely falling points $A$ and $B$ are in the gravitational field of an object with mass $M$. The initial velocity of the points is set to zero. Thus, the only change in $R$ will come from the acceleration that the points feel towards the mass. The distance of the first point from the mass $M$ is $R_0$ and the separation between the two points is $d$ initially. The trajectories of the two points (as shown in figure (\ref{fig:3})) are given by
%\begin{equation}
%	R_A(t) = R_0 - \frac{1}{2} \frac{M}{R_0^2} t^2,
%\end{equation}
%
%\begin{equation}
%	R_B(t) = R_0 +d - \frac{1}{2} \frac{M}{(R_0 + d)^2} t^2.
%\end{equation}
%
%At time $t = T_0$ a photon is emitted from point $A$ and reaches point $B$ at time $t = T$. We can calculate the time interval, $T-T_0$, using the metric in Eq (\ref{SCH}) with mass $M$. Setting $ds^2 = 0$ for the photons gives
%\begin{equation}
%	T-T_0 = (R_B(T_0 + d) - R_A(T_0))\left(1 - \frac{2M}{R_0} \right)^{-1}.	\label{11}
%\end{equation}
%
%Here we have assumed $M/R_0 \ll 1$. Note that $R_B$ is evaluated at time $T_0 + d$ since to leading order we know the time taken to go from $r_A$ to $r_B$ is $d$. We can convert the coordinate time interval $T- T_0$ into a proper time interval $\Delta \tau_0$ 
%
%\begin{equation}
%	\Delta \tau_0 = \left( 1 - \frac{2M}{r_0} \right)^{-\frac{1}{2}} (R_B(T_0 + d) - R_A(T_0)) = d + \mathcal{O} (M/R_0)	 \label{concept}
%\end{equation}
%Thus we see that for a weak gravitational field, $\tau$ is the distance $d$ to leading order. This shows that whenever we measure the proper time interval it takes for photons to travel a distance between two points, we are also measuring the distance between two points.  
%
%\begin{figure}[h]
%\begin{center}
%\includegraphics[width =\textwidth, height = 12cm]{time_concept.pdf}
%\caption{Space-time diagram showing two points free falling towards a mass $M$. A photon is emitted from the point on the inside at time $T_0$ and is received at the outer point at time $T$. The dark blue lines parametrize the trajectories of the two points and the orange line represents the trajectory of the photon. }
%\label{fig:3}
%\end{center}
%\end{figure}

\section{Memory effect in interferometer}

\subsection{Distance between $A$ and $B$}

\subsubsection{Defining trajectories}

An interferometer can measure the displacement between geodesics. There are two ways to calculate this displacement. The first is to directly calculate the photon geodesics from the metric and find the difference between them. The second is to integrate the geodesics deviation equation. In this paper we will use the first approach and calculate the trajectories of the interferometer points. We consider points $A$, $B$ and $C$ in figure \ref{fig:1} as the ends of the arms of the interferometer. 
We parametrize the trajectories of the points in a piecewise form. We will use the following terms to represent the acceleration and velocity of the points:

\begin{eqnarray}
	\bar{a}^{(A)}_{(2)} & = & \bar{a}^{(B)}_{(2)} = \bar{a}^{(C)}_{(2)} = \frac{M}{r_0^2}	\nonumber	\\
	a^{(A)}_{(2)} & = & a^{(B)}_{(2)} = a^{(C)}_{(2)} = \frac{M - \delta M}{r_0^2}	\nonumber	\\
	v^{(A)}_{(1)} & = & v^{(B)}_{(1)} = v^{(C)}_{(1)}  =  \frac{\delta M}{r_0} 	\nonumber	\\
	v^{(A)}_{(2)} & = & v^{(C)}_{(2)} =  \frac{\delta M^2}{r_0^2}	\nonumber	\\
	v^{(B)}_{(2)} & = & \frac{\delta M^2}{r_0^2} - \frac{\delta M d}{r_0^2}.
\end{eqnarray}
The notation is to take the superscript to be the label of the particle. The subscript represents the number of factors of $r_0$ the term is suppressed by, i.e $a^{(A)}_{(2)}$ is term of $\mathcal{O}(r_0^{-2})$. We will only keep terms up to order $\mathcal{O}(r_0^{-2})$.
The trajectory for point $A$ is

%\begin{equation}
%	r_A(t)  = \begin{cases} r_0 - \frac{1}{2} \frac{M}{r_0^2} t^2 & (t \leq 0), 	\\
%	 r_0 - \frac{1}{2} \frac{M-\delta M}{r_0^2} t^2 - \frac{\delta M }{r_0}\left( 1 + \frac{3}{2} \frac{\delta M}{r_0} \right) t \ & (t>0).
%	 \end{cases}	\label{rA}
%\end{equation}

\begin{eqnarray}
	r_A(\bar{t}) & = & r_0 - \frac{1}{2} \bar{a}_{(2)}^{(A)} \bar{t}^2, \hspace{5mm} (\bar{t} \leq 0)	\nonumber	\\
	r_A(t) & = & r_0 - \frac{1}{2} a^{(A)}_{(2)} t^2 - \left(v^{(A)}_{(1)} + v^{(A)}_{(2)} \right)t,  \hspace{5mm} (t>0)  \label{rA}.
\end{eqnarray}

\begin{itemize}

\item $t=0$ is defined to be the time when the shell crosses point $A$ and the distance $r_A(0) = r_0$. For $t<0$ we assume there is no velocity and the only term that contributes to the trajectory is the acceleration of point $A$ towards mass $M$. 

\item For $t>0$ there is still the acceleration term but the mass is different since the shell has carried $\delta M$ away. 

\item Along with that there is a term that corresponds to the velocity change (as we calculated from the IJC) coming from shell crossing. 
\end{itemize}
%
%\begin{equation}
%	r_B(t) = \begin{cases} r_0 + d  - \frac{1}{2}\frac{M}{(r_0+d)^2} t^2 &	(t \leq t_c), 	\\
%	 r_0 +d  - \frac{1}{2} \frac{M}{(r_0 + d)^2}t_c^2 - \frac{1}{2} \frac{M - \delta M}{(r_0 +d)^2}(t - t_c)^2 
%	- \frac{\delta M}{r_0 + d} \left( 1 + \frac{3}{2} \frac{\delta M}{r_0} \right) (t-t_c) -\frac{Md}{(r_0+d)^2}(t-t_c) &  (t>t_c).
%	\end{cases}	\label{rB}
%\end{equation}

The trajectory of point $B$ is given by

\begin{eqnarray}
	r_B(\bar{t}) & = &  r_0 + d  - \frac{1}{2} \bar{a}_{(2)}^{(B)} \bar{t}^2,  \hspace{5mm} (\bar{t} \leq \bar{t_c})	\nonumber	\\
	r_B(t) & = & r_0 +d  - \frac{1}{2}  \bar{a}^{(B)}_{(2)} t_c^2 - \left( \bar{a}_{(2)}^{(B)} t_c + v^{(B)}_{(1)} + v^{(B)}_{(2)} \right)(t-t_c) - \frac{1}{2} a_{(2)}^{(B)}(t-t_c)^2,  \hspace{5mm}  (t>t_c).	\label{rB}
\end{eqnarray}

\begin{itemize}

\item For $\bar{t} <\bar{t}_c$ there is just the acceleration, $\bar{a}^{(B)}_{(2)}$, of the point towards mass $M$. 

\item After $t>t_c$ the shell has crossed and now there is a change in distance in the time $t_c$ that the point is accelerating with ${a}^{(B)}_{(2)}$ towards $M-\delta M$.\footnote{Note that $\bar{t}_c$ and $t_c$ are different since they are coordinates in two different metrics.} This is the second term in the expression. 

\item The $\bar{a}^{(B)}_{(2)}$ term corresponds to the acceleration of the point towards mass $M-\delta M$. 

\item The terms with $v^{(B)}_{(1)} + v^{(B)}_{(2)}$ are the velocities picked up from the IJC. The $\bar{a}^{(B)}_{(2)} t_c$ term is the velocity gained while accelerating in the metric of mass $M$. 

\end{itemize}

\subsubsection{Relation between proper length and proper time}


The interferometer infers the distance between two points by measuring the proper time it takes for a photon to go from one end of the arm to the other and come back. We will show that \emph{this proper time $\tau$ is the same as twice the proper length $l$ up to $\mathcal{O}(r_0^{-2})$, as long as their is no relative velocity between the two points} (which in our case are the ends of the arms of the interferometer). The most general expression for the proper distance for a stationary observer is

\begin{equation}
	l_{AC} = \int \sqrt{g_{ab} dx^a dx^b}.	\label{proplen}
\end{equation}

Before shell crossing, the only difference between two points is their acceleration. Which means we can choose a constant (coordinate) time slice where the relative velocity of the two points is the same up to $\mathcal{O}(r_0^{-2})$.
In this section we calculate the proper distance before shell crossing $\bar{l}_{AB}$ between the two points $A$ and $B$ at time $\bar{t}$. Since we are only considering radial motion Eq (\ref{proplen}) simplifies to

\begin{equation}
	\bar{l}_{AB} = \int^{r_B(\bar{t})}_{r_A(\bar{t})} dr \ \bar{g}_{rr}^\frac{1}{2}  = \int^{r_B(\bar{t})}_{r_A(\bar{t})} dr \left( 1 + \frac{M}{r} + \frac{3}{2} \frac{M^2}{r^2}  + \mathcal{O}((M/r)^3) \right).
\end{equation}

In all the calculations we will only need terms of $\mathcal{O}(r_0^{-2})$. Thus we can use $r_B(\bar{t}) = r_0 + d$ and $r_A(\bar{t})  = r_0$. 

\begin{equation}
	 \bar{l}_{AB} =  \int^{r_B(\bar{t})}_{r_A(\bar{t})}  dr \ \bar{g}_{rr}^\frac{1}{2}  = d + M \ln \left( \frac{r_0 + d}{r_0} \right) - \frac{3}{2} M^2 \left( \frac{1}{r_0 + d} - \frac{1}{r_0} \right)
\end{equation}

This equation for $\bar{l}_{AB}$ is a general expression up to $\mathcal{O}(r_0^{-2})$.  It simplifies to 

\begin{equation}
	 \bar{l}_{AB} =  d \left( 1 + \frac{M}{r_0} - \frac{1}{2} \frac{Md}{r_0^2} + \frac{3}{2} \frac{M^2}{r_0^2} \right).
\end{equation}

We calculate the proper time $\bar{\tau}_{AB}$ in the appendix and notice that indeed $2\bar{l}_{AB} = \bar{\tau}_{AB}$. 

After shell crossing the two points will have no relative velocity between them (schematic diagram of the velocities is shown in figure \ref{velocity}) up to order $\mathcal{O}(r_0^{-2})$. Thus for the after shell crossing scenario we can also calculate the proper length $l_{AB}$ and know that it will be half the proper time $\tau_{AB}$. First lets show that there is no relative velocity between $A$ and $B$ explicitly. 

\subsubsection{Relative velocity between $A$ and $B$}

\begin{figure}[h]
\begin{center}
\includegraphics[width =\textwidth, height = 12cm]{vel.pdf}
\caption{Schematic of a spacetime diagram showing what happens to the velocities around shell crossing. The colour of the lines is defined in the same way as is done in figure \ref{calculation}. The new red lines represent the velocity of points $A$ and $B$. We show in Eqs (\ref{velA}, \ref{velB}) that the velocities of both points $A$ and $B$ are the same after shell crossing at time $t_c$. This comes from the fact that point $B$ is accelerating for some time in the metric of mass $M$. }
\label{velocity}
\end{center}
\end{figure}

We have assumed point $A$ has velocity zero at the time of shell crossing. After the shell crosses point $A$, it will pick up a velocity $\frac{\delta M}{r_0}$. In the time the shell takes to go from $A$ to $B$, $A$ will also gain a velocity from its acceleration towards the mass $M -\delta M$. The magnitude of the velocity gain from acceleration is $\frac{M-\delta M}{r_0^2} d$. The final velocity of $A$ at time $t_c$ is, therefore, given by

\begin{equation}
	v_A(t = t_c) = \frac{\delta M}{r_0} \left( 1 + \frac{\delta M}{r_0} \right)+ \frac{M-\delta M}{r_0^2} t_c.	\label{velA}
\end{equation}
Similarly, point $B$ will gain a velocity from the acceleration towards mass $M$ with magnitude $\frac{M}{r_0^2}d$. The velocity change from shell crossing is $\frac{\delta M}{r_0 + d}$. The final velocity of point $B$ at time $d$ is

\begin{equation}
	v_B(t=t_c) = \frac{\delta M}{r_0+d} \left( 1 + \frac{\delta M}{r_0} \right)+ \frac{M}{(r_0+d)^2} t_c.	\label{velB}
\end{equation}
We see that $v_A$ is the same as $v_B$ to $\mathcal{O}(r_0^{-2})$. This also implies that there shouldn't be any change in distance that grows with time at order $\mathcal{O}(r_0^{-2})$. 

The memory effect, $\Delta l_{AB}$ is the difference between $l_{AB}$ and $\bar{l}_{AB}$. In the appendix we have the full calculation of the proper length $l_{AB}$, here we quote the resulting memory effect. 

\begin{equation}
	\Delta l_{AB} \equiv l_{AB} - \bar{l}_{AB} = - \frac{\delta M^2 d}{r_0^2}.
\end{equation}
We find that indeed there is no time dependence in the change in proper length at $\mathcal{O}(r_0^{-2})$. Next we calculate an effect which is much more interesting since it grows with time.

\subsection{Distance between $A$ and $C$}

In this subsection we calculate the proper distance change between $A$ and $C$. First lets write down the trajectory equation for point $C$

%\begin{equation}
%	r_{C}(t)_\bot = \begin{cases} d - \frac{1}{2} \frac{M}{(r_0^2 + d^2)} t^2 \sin \theta ,  & (t\leq 0 )	
%	\\ d - \frac{1}{2} \frac{M - \delta M}{(r_0^2 + d^2)} t^2 \sin \theta - \frac{\delta M}{\sqrt{r_0^2 + d^2}} t \sin \theta,  & (t >0)  \end{cases}. 
%\end{equation}

\begin{eqnarray}
	r_{C}(\bar{t})  & = & d - \frac{1}{2} \bar{a}_{(2)}^{(C)}  \bar{t}^2 ,  \hspace{5mm} (\bar{t} \leq 0 )		\nonumber	\\
	r_{C}(t) & = &  d - \frac{1}{2} a_{(2)}^{(C)} t^2 - \left(v^{(C)}_{(1)} + v^{(C)}_{(2)} \right) t,  \hspace{5mm} (t >0) . 
\end{eqnarray}

From the geometry in figure \ref{fig:1} we see that only $\theta$ and $r$ will change in the motion of $A$ and $C$. Thus we can drop the $\phi$ dependence from the start and write the above integral as
\begin{equation}
	 l_{AC} = \int^{\theta_f}_{\theta_i} d \theta \ g_{\theta \theta}^\frac{1}{2} \left(  1 + \underbrace{(g_{rr}/g_{\theta \theta}) (dr/d\theta)^2}_{T3}  \right)^\frac{1}{2}.
\end{equation}
where $\theta_f = d/r_0$ and $\theta_i = 0$. To evaluate $T3$ we would need the geodesic equation for $r$ as function of $\theta$. Instead of evaluating it explicitly we note that $\frac{dr}{d \theta}$ will be of order $d$. Since $g_{\theta \theta} = \frac{1}{r_0^2} + \mathcal{O}(r_0^{-3})$, to leading order we can ignore the effect of $T3$. Integrating the first term gives

\begin{equation}
	l_{AC} = r_C(t) (\theta_f - \theta_i) = r_C(t) \frac{d}{r_0}.
\end{equation}
Before shell crossing the proper length $\bar{l}_{AC} = d + \mathcal{O}(r_0^{-3})$. Whereas after shell crossing, there will be an additional velocity kick term in the proper length, $l_{AC}$, which will give a time dependent term at $\mathcal{O}(r_0^{-2})$

\begin{equation}
	l_{AC} = d - \frac{\delta M d}{r_0^2} t.
\end{equation}

The difference in proper lengths will be 

\begin{equation}
	\Delta l_{AC} \equiv l_{AC} - \bar{l}_{AC} = - \frac{\delta M d}{r_0^2} t.
\end{equation}


%
%\subsection{The effect of change in velocity on photons}
%
%%
%%\begin{figure}[h]
%%\begin{center}
%%\includegraphics[width=\textwidth,height=10cm]{Lens.pdf}
%%\caption{Geometry of the astrophysical interferometer formed by pulsar scintillometry. Due to scattering or lensing, the image we see is an interference pattern of two light rays represented by the blue lines. If the separation of the two light rays has a component along the longitudinal (radial) direction from the SN, the spacetime distortion of the neutrino shell will change the interference pattern we see. We draw the lens to be behind the SN, but it could have been in front of it and the effect is the same.}
%%\label{fig:4}
%%\end{center}
%%\end{figure}
%
%An interferometer works by sending photons from a source to another point that reflects them back to the source. Usually this is done using two arms of equal length so that if the length of one of the arms increases the photons that leave the source will not arrive back in phase with the photons that went along the other arm (as there will be a path difference in their travel distance) and it is this phase change that is observed in an interferometer. There is an equivalent way of calculating this effect which is to count the number of wavelengths of the photon along its total trajectory. In general, a photon trajectory, $k^\mu$, is of the form
%
%\begin{equation}
%	k^\mu = (E, p_r, p_\theta, p_\phi),
%\end{equation}
%for a photon moving in a radial direction $p_r = \frac{1}{\lambda}$ where $\lambda$ is the wavelength of the photon. A photon moving in a Schwarzschild geometry will have the following four vector 
%
%\begin{equation}
%	k^\mu(\omega_\infty, M, r) = \omega_\infty \left(  \frac{1}{1 - \frac{2M}{r}}, \sqrt{1 - \frac{b^2}{r^2} \left( 1 - \frac{2M}{r} \right)}, \frac{b}{r^2}, 0 \right),
%\end{equation}
%$\omega_\infty$ is a normalization factor that corresponds to the frequency of the photon as observed by an observer at rest sitting infinitely far away and $b$ represents the impact parameter of the photon. By setting $b= 0$ we get
%
%\begin{equation}
%	k^\mu(\omega_\infty, M, r) = \omega_\infty \left( \frac{1}{1 - \frac{2M}{r}}, 1 , 0 , 0  \right).
%\end{equation}
%This photon has wavelength $\lambda_\infty = \frac{1}{\omega_\infty}$. As an illustration, we consider the effect a neutrino shell has on just one arm of an interferometer, i.e two free falling points separated by distance $r_a - r_b = d$,\footnote{This is representative of the two points $A$ and $B$ in figure \ref{fig:1} as the two points are assumed to be in line with the SN.} as shown in the bottom part of figure \ref{fig:2}. A photon leaves from $r_a$ with four vector $k^\mu_i(\omega_a)$ and goes to point $r_b$. The frequency measured, $\omega_{O1}$, by an observer at $r_b$ of this photon is
%
%\begin{equation}
%	\omega_{O1} = - g_{\mu \nu}(M_i, r_b) k^\mu(M_i, \omega_a, r_b) \zeta^\nu_i(M_i, r_b).	\label{9}
%\end{equation}
%This must be the same as the frequency observed for a photon coming from the opposite direction which, in figure \ref{fig:2}, is denoted by $k^\mu_i(\omega_b)$, 
%
%\begin{equation}
%	\omega_{O1} = - g_{\mu \nu}(M_i, r_b) k^\mu(M_i, \omega_b, r_b) \zeta^\nu_i(M_i, r_b).	\label{10}
%\end{equation}
%Since the photon is travelling in the opposite direction in this case, the velocity component of the $k^\mu_i(\omega_b)$ will have the opposite sign
%
%\begin{equation}
%	k^\mu(\omega_b) = \omega_b \left( \frac{1}{1 - \frac{2M}{r}}, -1, 0, 0 \right).
%\end{equation}
%By equating Eq (\ref{9}) and (\ref{10}) we see $\omega_a = \omega_b$ and therefore the wavelengths of the photons must be the same. Since the distance between the two points, $d$, does not change, the total number of wavelengths traversed by the photons along there paths is a constant number, $N_i$, given by
%
%\begin{equation}
%	N_i = \frac{d}{\lambda_a} = d\omega_a = d \omega_b
%\end{equation}
%When a neutrino shell crosses the two points, they pick up a velocity as is shown in Eq (\ref{5}). This means both, the frequency of the photons and the distance between the two points will change. Let's begin by calculating the change in the frequency. Consider the photon $k^\mu_f(\omega_1)$ emitted from a distance $r_1$, shown in the upper part of figure \ref{fig:2}. An observer $O2$ at $r_2$ will measure the frequency of photon $k^\mu_f(\omega_1)$ to be
%
%\begin{eqnarray}
%	\omega_{O2} & = & -g_{\mu \nu}(M_f, r_2) k^\mu(M_f, \omega_1, r_2) \zeta^\nu(M_f, r_2) 	\nonumber	\\
%	& = & \left( 1 + \frac{M_f + \delta M}{r_2} + \frac{2M_f \delta M}{r_2^2} \right) \omega_1.
%\end{eqnarray}
%Similarly the photon $k^\mu_f(\omega_2)$ has the measured frequency
%\begin{eqnarray}
%	\omega_{O2}  & = & -g_{\mu \nu}(M_f, r_2) k^\mu(M_f, \omega_2, r_2) \zeta^\nu(M_f, r_2) 	\nonumber	\\
%	& = & \left( 1 + \frac{M_f - \delta M}{r_2} - \frac{2M_f \delta M}{r_2^2} \right) \omega_2.
%\end{eqnarray}
%Combining these two equations gives, to second order in small quantities, 
%
%\begin{equation}
%	\omega_1 = \left( 1- \frac{2 \delta M}{r_2} - \frac{4 M_f \delta M - M_f^2 + \delta M^2}{r_2^2} \right) \omega_2.
%\end{equation}
%The distance, $d$, will also change as both the points will have different velocities. Therefore the distance, $d_1$, traversed by photon $k^\mu_f(\omega_1)$ is
%
%\begin{equation}
%	d_1 = r_2 - r_1 = d - \frac{\delta M}{r_b} t_2 + \frac{\delta M}{r_a} t_1.
%\end{equation}
%We want to get everything in terms of one time $t_1$, so lets find $t_2$ in terms of $t_1$
%
%\begin{equation}
%	t_2 = t_1 + d -\frac{\delta M}{r_a} t_1 + \frac{\delta M}{r_b} t_2
%\end{equation}
%
%\begin{equation}
%	t_2 \left( 1 - \frac{\delta M}{r_b} \right) = t_1 + d - \frac{\delta M t_1}{r_a}
%\end{equation}
%which to leading order is
%
%\begin{equation}
%	t_2 = t_1 + d + \mathcal{O}(r^{-1}).
%\end{equation}
%and so $d_1$ becomes,
%
%\begin{equation}
%	d_1 = d\left(1 - \frac{\delta M }{r_a} - \frac{\delta M t_1 }{r_a^2} + \mathcal{O}(r_a^{-3}) \right).
%\end{equation}
%Therefore the number of wavelengths, $N_{f1}$, traversed by photon $k^\mu_f(\omega_1)$ is
% 
% \begin{eqnarray}
% 	N_{f1} & = & \frac{d_1}{\lambda_1} = \frac{d\left(1 - \frac{\delta M }{r_a} - \frac{\delta M t_1 }{r_a^2} \right)}{\lambda_1}
%\end{eqnarray}
%So we see that there is a term that increases linearly with time and so the number of wavelengths increases linearly with time. Similarly we can calculate the number of wavelengths, $N_{f2}$, traversed by the photon $k^\mu_f (\omega_2)$ on its way back to $r_3$. The distance $d_3$ traveresed by the photon is
%
%\begin{equation}
%	d_3 = r_2 - r_3 = d - \frac{\delta M}{r_a} t_3 + \frac{\delta M}{r_b} t_2
%\end{equation}
%
%Again lets get $t_3$ in terms of $t_2$ (which we know how to write in terms of $t_1$), 
%
%\begin{equation}
%	t_3 = t_2 + d + \frac{\delta M}{r_a} t_3 - \frac{\delta M}{r_b} t_2
%\end{equation}
%
%\begin{equation}
%	t_3 \left( 1 - \frac{\delta M}{r_a} \right) = t_2 + d - \frac{\delta M }{r_b} t_2
%\end{equation}
%
%and so to leading order
%
%\begin{equation}
%	t_3 = t_2 + d = t_1 + 2d + \mathcal{O}(r^{-1})
%\end{equation}
%
%which is again expected. So $d_3$ becomes
%
%\begin{eqnarray}
%	d_3 & = & d - \frac{\delta M d}{r_a} - \frac{\delta M t_1}{r_a} + \frac{\delta M}{r_b} t_1	\nonumber	\\
%	& = & d \left( 1 - \frac{\delta M}{r_a} \right) + \frac{\delta M d}{r_a^2} t_1
%\end{eqnarray}
%
%And the number of wavelengths $N_{f2}$ is
%
%\begin{equation}
%	N_{f2} = \frac{d \left( 1 - \frac{\delta M}{r_a} \right) + \frac{\delta M d}{r_a^2} t_1}{\lambda_2}
%\end{equation}
%
%where $\lambda_2 = \frac{1}{\omega_2}$.
%

%%%%%%%%%%%%%%%%%%%%%%%%%%%%%%%%%%
%
%\subsection{Distance change in interferometer}
%
%We have seen in the previous section how the effect of the nuetrino shell crossing the points manifests itself to an interferometer. Here we expand on this calculation and apply it to a general interferometer like the one shown in figure \ref{fig:1}. 
%\\
%\\
%The relative velocity between the ends of a horizontal interferometer, $A$ and $B$, that is in radial alignment with the SN as shown in figure \ref{fig:1} is given by
%\begin{equation}
%	\Delta L_{AB} =  (v_A - v_B)t = d \frac{\delta M}{r_E^2}  t+ \mathcal{O}(r_E^{-3})~.
%\end{equation}
%where $v_A$ and $v_B$ are the velocities picked up by points $A$ and $B$ after shell crossing and $t$ is the time passed after shell crossing. Note that since the point $A$ picks up a larger velocity toward the supernova, $\Delta L_{AB}$ is \emph{increasing}.
%
%The total velocities for points $B$ and $C$ have the same magnitudes but are pointing in slightly different directions. It is easy to work out the geometry to see that
%\begin{eqnarray}
%	\Delta L_{BC} &=& -\left( \frac{ \delta M}{r_E} \sin \theta \right) t 
%	\\ \nonumber
%	&=& - d \frac{\delta M}{r_E^2} t + \mathcal{O}(r_E^{-3})~.
%\end{eqnarray}
%Here $\theta\approx (d/r_E)$ is the angle between point $B$ and $C$ to the SN. Since they both fall toward the SN, they are getting closer to each other thus $\Delta L_{BC}$ is \emph{decreasing}\footnote{This is important; if the distance $\Delta L_{BC}$ was also increasing, since the magnitude of the increase is the same as $\Delta L_{AB}$, we would not see an effect since the photons in the interferometer arms would still have the same phase when they come back to point $B$.}.
%
%In summary, we have an interferometer whose one arm decreases in length while the other increases, resulting in a detectable change in the interference pattern. 
%\begin{center}
%\begin{tabular}{| c | c |} 
%\hline
%$\Delta L_{AB}$ &  $  \Delta L_{BC} $  \\   \hline 
%$\frac{\delta M d}{r_E^2} t$ & $ -\frac{\delta M d}{r_E^2} t$  \\ 
%\hline 
%\end{tabular}
%\end{center}
%Note that for the conventional memory effect, the signal is maximized when the interferometer is face-on to the source. In our case, a face-on interferometer would get zero signal, since both arms will be decreasing in length. Our signal is maximized by having a longitudinal arm, in this case $AB$.

%\section{Change in proper time}
%\label{Ptime}
%Interferometers like LIGO measure a change in the phase of light. The phase takes the form $\omega \tau$ where $\omega$ is the frequency of the photon and $\tau$ is the proper time traversed by the photon clock. In \cite{Pulsar_acc} we showed that there is no change in frequency in the case when the objects are aligned, so that would be in the case when the interferometer is oriented radially such as $AB$. For simplicity lets just look at the radial case of $AB$. Since there is no change in frequency, the change in phase will be given by the change in proper time (which is also intuitively pleasing as it corresponds to a change in a physical quantity which is the proper length between the two points). 
%\begin{figure}[h!]
%\begin{center}
%\includegraphics[scale = 0.4]{shellcrossing.pdf}
%\caption{This is showing the change in proper time before and after shell crossing for an interferometer in the radial direction of the SN, so that would correspond to the points $A$ and $B$ in figure \ref{fig:1}. The blue lines represent the motion of photons and the orange line is the neutrino shell.}
%\label{fig:2}
%\end{center}
%\end{figure}
%By carefully expanding the equations of motion of a radial photon in a Schwarzschild metric one can find the expressions for proper time before, $\Delta \tau_i$, and after, $\Delta \tau_f$, shell crossing. The full calculation is presented in the appendix, the result we are interested in is the difference in the proper times, $\Delta \tau \equiv \Delta \tau_i - \Delta \tau_f$, and it is given by
%\begin{equation}
%	\Delta \tau=  \frac{2d}{r_E^2} ( \frac{3}{2} (M_i^2 - M_f^2) - \frac{d \delta M}{2} - 3 M_f \delta M - \frac{\delta M^2}{2} \right - t \delta M  ).
%\end{equation}
%This shows that there is no change in proper time to $\mathcal{O}(r_E^{-1})$ which is what is expected and there is a term that grows linearly with time at $\mathcal{O}(r_E^{-2})$ which is in agreement with results presented in section \ref{RelV}.

\section{Observation with Space-based Interferometers}
\label{obs}
Taking the generic form of the change in distance as $\Delta L \sim \frac{\delta M d}{r_0^2} t$ we can estimate the distance a SN would have to be from the interferometer to have an observable change in strain, which is a unitless number quantifying the amount of space-time distortion.
\begin{equation}
h \sim \frac{\Delta L}{d} \sim \frac{\delta M t}{r_0^2}~.
\label{eq-strain}
\end{equation} 
Since our strain grows linearly with time, we do not expect detections from ground based experiments as for those setups the three points $A,B,C$ cannot remain in free fall for a long enough amount of time such that the signal builds up to an observable value. 

If we plot our effect on the strain-frequency diagram \cite{GWcurves} that is usually used to compare different interferometers, it will be a 45-degree line. Thus the first point at which the sensitivity curve of a device crosses with a 45-degree line will give the best chance for our effect being detected. In all these estimations, we take $\delta M$ to be a faction of a solar mass, and take the corresponding Schwarzschild radius to be $1~km$ for simplicity.

For LISA, the best observing frequency is $\sim 0.5 \times10^{-2} Hz$ with a sensitivity in strain $\sim 10^{-21}$. Using Eq.~(\ref{eq-strain}), we can solve for the distance to the SN, $r_0$, for our effect to be detectable.
\begin{eqnarray}
	r_0 & = &  \left(  \frac{\delta M}{h} t \right)^\frac{1}{2} \label{Meas}	\\
	& = &  \left( \frac{1 \ km}{10^{-21}} \times 10^7 \ km \right)^\frac{1}{2} \approx 10^{14} \ km = 10 \ ly.
\end{eqnarray}
This is clearly too close. It has been estimated that only once in $10^8$ years will a SN go off within a distance of $30 \ ly$ \cite{EllSch93}.\footnote{And if that happens, it might kill us.} By a naive volume scaling, an explosion within 10 $ly$ only occurs once every billion years.

If we look at the Big Bang Observer (BBO) instead, the best observing frequency is $\sim 0.5 Hz$ with a sensitivity in strain $\sim 10^{-24}$. First of all, this frequency range does not have as many background signals from compact binaries, making it a much better device to measure our effect. The improved sensitivity gives a value for $r_0$ of $\sim 100 \ ly$. This is a factor of $10^3$ increase in the volume for detectable events, thus improves the expectation of one SN that is within $10 \ ly$ to occur in less than a million years . That is unfortunately still a long shot.

In this type of simple estimation, we cannot go lower in the frequency. The exact duration of the neutrino-shell passage is not known, but we do no expect it to be much less than a second. Thus for higher frequencies, the co-dimension-one delta function approximation breaks down, and the effect will be weaker than Eq.~(\ref{eq-strain}).

Finally, we expect 2 to 3 SN per century in our galaxy and we can assume that the next SN would be at a distance comparable to the galactic diameter of $\sim 10^5 \ ly$. If we are going to measure such effect at $1 \ Hz$, Again using Eq.~(\ref{eq-strain}), we find
\begin{equation}
	h = \frac{ 1~km \times ~ (3\times10^8~m)}{(10^5~ly)^2}\sim 10^{-30}~.
\end{equation}
This requires a measurement of the strain that is six orders of magnitude better than BBO and is not yet achievable by interferometers that are currently being planned. 


\acknowledgments

This work is supported by the Canadian Government through the Canadian Institute for Advance Research and Industry Canada, and by Province of Ontario through the Ministry of Research and Innovation.

\appendix

%
%\section{Full calculation of proper time}
%
%\subsection{Distance between $B$ and $C$}
%
%In this calculation we first calculate the time, $t_1$, it takes for the photon released at time $t$ from point $A$ to reach point $B$. Then we calculate the time, $t_2$, it takes for the photon to come back to point $A$. The quantities are clearly defined in figure \ref{calculation}. The final result is obtained by adding these times and converting to proper time. 
%\\
%\\
%Lets start by finding $t_1$. The expression that will give us a value for $t_1$ is the metric in Eq (\ref{SCH}) with $ds^2 = 0$. To begin with we look at the leading order calculation. The leading order behaviour of $t_1$ will allow us to make simplifying approximations in the higher order calculations. 
%
%\begin{eqnarray}
%	t_1  & = &  \left(  1 - \frac{2(M- \delta M)}{r_0} \right)^{-1} (r_B(t+t_1) - r_A(t))	\nonumber	\\
%	& = &  d - \frac{\delta M t_1}{r_0} + \frac{\delta M d}{r_0} + \frac{2(M - \delta M)}{r_0}d
%\end{eqnarray}
%
%Note that the term on the r.h.s containing the $t_1$ can be moved over to the l.h.s to solve for $t_1$
%
%\begin{equation}
%	t_1 \left( 1 + \frac{\delta M}{r_0} \right) = d \left( 1 + \frac{\delta M}{r_0} + \frac{2(M - \delta M)}{r_0} \right)
%\end{equation}
%
%Simplifying this gives the final expression for $t_1$
%
%\begin{equation}
%	t_1 = d \left( 1 + \frac{2(M - \delta M)}{r_0} \right) 	\label{A3}
%\end{equation}
%
%Now lets work out $t_2$. 
%
%\begin{eqnarray}
%	t_2 & = & \left( 1 - \frac{2(M - \delta M)}{r_0} \right)^{-1} ( r_B(t + t_1) - r_A(t + t_1 + t_2) )	\nonumber \\
%	& = & d + \frac{\delta M}{r_0} d + \frac{\delta M}{r_0} t_2 + \frac{2(M- \delta M)}{r_0} d
%\end{eqnarray}
%
%Move the $t_2$ term on the r.h.s to the l.h.s
%\begin{equation}
%	t_2 \left( 1 - \frac{\delta M}{r_0} \right) = d \left( 1+  \frac{\delta M}{r_0} + \frac{2(M - \delta M)}{r_0} \right)  
%\end{equation}
%
%Simplifying to the final expression
%
%\begin{equation}
%	t_2 = d \left( 1 + \frac{2(M - \delta M)}{r_0} + \frac{2 \delta M}{r_0} \right)
%\end{equation}	
%
%Now we combine them to get the total coordinate time it takes for the photon to go from point $A$ to $B$ and come back. 
%
%\begin{equation}
%	 t_1 + t_2 = 2d \left( 1 + \frac{2(M - \delta M)}{r_0} + \frac{\delta M}{r_0} \right)
%\end{equation}
%
%Converting to proper time, $\tau$, seen by point $A$
%
%\begin{eqnarray}
%	\tau & = & - \left( 1 - \frac{2(M - \delta M)}{r_0} \right)^\frac{1}{2} (t_1 + t_2)\nonumber	\\
%	& = & -2d \left( 1 + \frac{M}{r_0} \right)
%\end{eqnarray}
%
%Since we are interested in the difference in proper time before and after shell crossing, we need to compute the proper time without shell crossing. We do this using the same method. 
%We start by finding $\bar{t}_1$.
%
%\begin{equation}
%	\bar{t}_1  = \left( 1 - \frac{2M}{r_0} \right)^{-1}(\bar{r}_b(t + t_1) - \bar{r}_A(t) ) = d \left( 1 - \frac{2M}{r_0} \right)^{-1}
%\end{equation}
%
%$\bar{t}_2$ is the same as $\bar{t}_1$ to leading order. The proper time $\bar{\tau}$ is
%
%\begin{equation}
%	\bar{\tau} = - \left( 1 - \frac{2M}{r_0} \right)^\frac{1}{2} (\bar{t}_1 + \bar{t}_2) = -2d \left( 1 + \frac{M}{r_0} \right)	\label{A11}
%\end{equation}
%
%The difference between the proper times, to $\mathcal{O}(r_0^{-1})$ is 
%
%\begin{equation}
%	\Delta \tau \equiv \tau - \bar{\tau} =  0.	\label{M1}
%\end{equation}
%
%Now we look at the $\mathcal{O}(r_0^{-2})$ calculation. 
%
%\begin{eqnarray}
%	t_1 \left( 1 - \frac{2(M - \delta M)}{r_0} \right) & = & d - \frac{1}{2} \frac{M d^2}{(r_0 + d)^2} - \frac{1}{2} \frac{(M - \delta M)}{(r_0 + d)^2 } (t + t_1 -d)^2 - \frac{\delta M}{r_0} (t  + t_1 -d) + \frac{\delta Md}{r_0^2}(t + t_1 -d) 	\nonumber	\\
%	& - & \frac{Mdt}{r_0^2} + \frac{1}{2} \frac{M - \delta M}{r_0^2} t^2 + \frac{\delta M}{r_0}t
%\end{eqnarray}
%
%We previously showed in Eq (\ref{A11}) that to order $\mathcal{O}(r_0^{-1})$ we don't expect any change in proper time and also in Eq (\ref{A3}) we show that $t_1 -d = \mathcal{O}(r_0^{-1})$. Therefore we can set the $t_1 - d$ in the equation above to zero as it is multiplying a term that is $\mathcal{O}(r_0^{-2})$ and we are not interested in terms at order $\mathcal{O}(r_0^{-3})$. 
%
%\begin{equation}
%	t_1 \left( 1 - \frac{2(M - \delta M)}{r_0} \right) = d - \frac{1}{2} \frac{Md^2}{r_0^2} - \frac{\delta M}{r_0} t_1 + \frac{\delta M d}{r_0} - \frac{(M - \delta M) dt}{r_0^2} 
%\end{equation}
%
%Moving the $t_1$ term to the left
%
%\begin{equation}
%	t_1\left( 1 - \frac{2(M - \delta M)}{r_0} + \frac{\delta M}{r_0} \right) = d \left( 1 - \frac{1}{2} \frac{Md}{r_0^2} + \frac{\delta M}{r_0} - \frac{(M - \delta M)t}{r_0^2} \right)
%\end{equation}
%
%Finally simplifying
%
%\begin{eqnarray}
%	t_1 & = & d \left( 1 + \frac{2(M - \delta M)}{r_0} - \frac{1}{2} \frac{M d}{r_0^2} - \frac{( M - \delta M)}{r_0^2}t + \frac{4(M - \delta M)^2}{r_0^2} - \frac{2 \delta M(M - \delta M)}{r_0^2} \right)
%\end{eqnarray}
%
%Here we note that when $\delta M =0$ this equation reduces to Eq (\ref{A21}). 
%\\
%\\
%Now lets compute $t_2$
%
%\begin{eqnarray}
%	t_2 \left( 1 - 2 \frac{M - \delta M}{r_0} \right) & = & r_B(t + t_1) - r_A(t+t_1 + t_2)	\nonumber	\\
%	& = & d - \frac{1}{2} \frac{Md^2}{r_0^2} - \frac{1}{2} \frac{(M - \delta M)}{r_0^2} t^2 - \frac{\delta M}{r_0}  \left( 1 - \frac{M- \delta M}{r_0} \right) (t + t_1 -d) + \frac{\delta Mdt}{r_0^2} - \frac{Md t}{r_0^2}	\nonumber	\\
%	& + & \frac{1}{2} \frac{(M - \delta M)(t+t_1 + t_2)^2}{r_0^2} + \frac{\delta M}{r_0}  \left( 1 - \frac{M- \delta M}{r_0} \right) (t + t_1 + t_2)	\nonumber	\\
%	& = & d - \frac{1}{2} \frac{M}{r_0^2} d^2 + \frac{\delta M d}{r_0} + \frac{\delta M t_2}{r_0} + \frac{(M - \delta M)dt}{r_0^2} + \frac{2(M - \delta M)d^2}{r_0^2} - \frac{2 \delta M( M - \delta M)d}{r_0^2} 
%\end{eqnarray}
%
%Again we set $t_1, t_2 = d$ as the corrections to this will come in at $\mathcal{O}(r_0^{-3})$ because the terms that contain $t_1, t_2$ are already $\mathcal{O}(r_0^{-2})$. 
%\\
%Moving the $t_2$ term to the l.h.s 
%
%\begin{eqnarray}
%	t_2 \left( 1 - \frac{2(M - \delta M)}{r_0} - \frac{\delta M}{r_0} \right) & = & d \left( 1 + \frac{\delta M}{r_0} - \frac{1}{2} \frac{Md}{r_0^2} + \frac{\delta M t}{r_0^2} + \frac{(M - \delta M)t}{r_0^2} + \frac{2(M - \delta M)d}{r_0^2} - \frac{2 \delta M(M - \delta M)}{r_0^2}  \right)	\nonumber	\\
%\end{eqnarray}
%
%Finally simplifying
%
%\begin{eqnarray}
%	t_2 & = & d \left( 1 + \frac{\delta M}{r_0} - \frac{1}{2} \frac{Md}{r_0^2} + \frac{(M - \delta M) t}{r_0^2} + \frac{2(M - \delta M) d}{r_0^2} - \frac{2\delta M(M - \delta M)}{r_0^2} \right)	\nonumber	\\
%	& \times & \left( 1 + \frac{2(M - \delta M)}{r_0} + \frac{\delta M}{r_0} + \left(\frac{2(M - \delta M)}{r_0} + \frac{\delta M}{r_0} \right)^2 \right)		\nonumber	\\
%	& = & d \left( 1+ \frac{2 \delta M}{r_0} + \frac{2(M - \delta M)}{r_0} - \frac{1}{2} \frac{Md}{r_0^2} + \frac{(M - \delta M)t}{r_0^2} + \frac{2(M - \delta M) d}{r_0^2} + \frac{ 6 \delta M (M - \delta M)}{r_0^2} + \frac{2\delta M^2}{r_0^2} \right.	\nonumber	\\
%	& + & \left. \frac{4(M - \delta M)^2}{r_0^2}  - \frac{2(M - \delta M) \delta M}{r_0^2} \right)	\nonumber	\\
%	& = & d \left( 1+ \frac{2 \delta M}{r_0} + \frac{2(M - \delta M)}{r_0} - \frac{1}{2} \frac{Md}{r_0^2} + \frac{(M - \delta M)t}{r_0^2} + \frac{2(M - \delta M) d}{r_0^2} + \frac{ 4 \delta M (M - \delta M)}{r_0^2} + \frac{2\delta M^2}{r_0^2} \right.	\nonumber	\\
%	& + & \left. \frac{4(M - \delta M)^2}{r_0^2}  \right)	\nonumber	\\
%\end{eqnarray}
%
%as a check again we see that this reduces to $\bar{t}_2$ when $\delta M = 0$. Adding the two times to get the coordinate time $t_1 + t_2$,  
%
%\begin{eqnarray}
%	t_1 + t_2 & = & 2d \left( 1 + \frac{\delta M}{r_0} + \frac{2(M - \delta M)}{r_0} - \frac{1}{2} \frac{Md}{r_0^2} + \frac{4(M - \delta M)^2}{r_0^2} + \frac{(M - \delta M)d}{r_0^2} + \frac{\delta M (M - \delta M)}{r_0^2} + \frac{\delta M^2}{r_0^2} \right)
%\end{eqnarray}
%
%converting to proper time
%
%\begin{eqnarray}
%	\tau & = & - \left( 1 - \frac{2(M - \delta M)}{r_0} \right)^\frac{1}{2}(t_1 + t_2)	\nonumber	\\
%	& = & -2d \left( 1 - \frac{(M - \delta M)}{r_0} - \frac{1}{2} \frac{(M - \delta M)^2}{r_0^2} \right) 	\nonumber	\\
%	& \times &\left( 1 + \frac{\delta M}{r_0} + \frac{2(M - \delta M)}{r_0} - \frac{1}{2} \frac{M d}{r_0^2} + \frac{4(M - \delta M)^2}{r_0^2} + \frac{(M - \delta M)d}{r_0^2} + \frac{\delta M(M - \delta M)}{r_0^2} + \frac{\delta M^2}{r_0^2} \right)	\nonumber	\\
%	& = & -2d \left(1 + \frac{M}{r_0} - \frac{1}{2} \frac{Md}{r_0^2} + \frac{3}{2} \frac{(M - \delta M)^2}{r_0^2} + \frac{(M - \delta M)d}{r_0^2} + \frac{\delta M^2}{r_0^2} \right)
%\end{eqnarray}
%	
%Lets compare this to the proper time before shell crossing $\bar{\tau}$. Lets start with $t_1$ 
%
%\begin{eqnarray}
%	\bar{t}_1 & = & \left( 1 - \frac{2M}{r_0} \right)^{-1} (r_B(\bar{t} + \bar{t}_1) - r_A(\bar{t}))	\nonumber	\\
%	& = & \left( 1 - \frac{2M}{r_0} \right)^{-1} \left( d - \frac{1}{2} \frac{M}{r_0^2} (\bar{t} + \bar{t}_1)^2 + \frac{1}{2} \frac{M}{r_0^2} \bar{t}^2 \right)	\nonumber	\\
%	& = & d \left( 1 + \frac{2M}{r_0} + \frac{4M^2}{r_0^2} - \frac{M \bar{t}}{r_0^2} - \frac{1}{2} \frac{Md}{r_0^2} \right)	\label{A21}
%\end{eqnarray}
%
%Now $t_2$ 
%
%\begin{eqnarray}
%	\bar{t}_2 & = & \left( 1 - \frac{2M}{r_0} \right)^{-1} \left(d + \frac{1}{2} \frac{M}{r_0^2} (\bar{t}^2 + 4 d \bar{t} + 4d^2 - \bar{t}^2 - 2d \bar{t}  - d^2) \right)	\nonumber	\\
%	& = & d \left( 1 + \frac{2M}{r_0} + \frac{4M^2}{r_0^2} + \frac{M\bar{t}}{r_0^2} + \frac{3}{2} \frac{Md}{r_0^2} \right)
%\end{eqnarray}
%
%The proper time is
%
%\begin{eqnarray}
%	\bar{\tau} & = & - \left( 1 - \frac{2M}{r_0} \right)^{\frac{1}{2}}(\bar{t}_1 + \bar{t}_2) 	\nonumber	\\
%	& = & - 2d \left( 1  +\frac{M}{r_0} + \frac{3}{2} \frac{M^2}{r_0^2} + \frac{1}{2} \frac{Md}{r_0^2} \right)
%\end{eqnarray}
%
%The difference in proper times is therefore given by 
%
%\begin{eqnarray}
%	\tau - \bar{\tau} & = & -2d \left( - \frac{Md}{r_0^2} + \frac{3}{2} \frac{(M - \delta M)^2}{r_0^2} + \frac{(M - \delta M)d}{r_0^2} + \frac{\delta M^2}{r_0^2} - \frac{3}{2}\frac{M^2}{r_0^2} \right)		\nonumber	\\
%	& = & -2d \frac{\delta M}{r_0^2} (-3M + \frac{5}{2} \delta M -d)
%\end{eqnarray}
%	
%%	
%%\subsection{Change in frequency}
%%
%%Since the phase of the photons is a product of the proper time and the frequency $\omega$. The frequency at time $t$ is $\omega_0$
%%
%%\begin{eqnarray}
%%	\omega_0 & =& -  k^\mu \zeta_f^\nu g_{\mu \nu} |_{r_A(t)}		\nonumber	\\
%%	& = & - k^0 \zeta_f^0 g_{00}|_{r_A(t)} - k^1 \zeta_f^1g_{11}|_{r_A(t)}	\nonumber	\\
%%	& = & \omega_\infty \left( \left(1 - \frac{2(M - \delta M)}{r_A(t)} \right)^{-\frac{1}{2}} + \frac{\delta M}{r_0} \left(1 - \frac{2(M - \delta M)}{r_A(t)} \right)^{-1} \right)	\nonumber	\\
%%	& = & \omega_\infty \left( 1 + \frac{M}{r_A(t)} - \delta M \left( \frac{1}{r_A(t)} - \frac{1}{r_0} \right) \right)
%%\end{eqnarray}
%%
%%This will be observed at $B$ at time $t+t_1$
%%
%%\begin{equation}
%%	\omega_1 = \omega_\infty \left( \left( 1 - \frac{2(M - \delta M)}{r_B(t+t_1)} \right)^{-\frac{1}{2}} + \frac{\delta M}{r_0} \left( 1 - \frac{2(M - \delta M)}{r_B(t+t_1)} \right)^{-1} \right)
%%\end{equation}
%%
%%The point $B$ now emits a photon with frequency $\omega_1$ and the measured frequency at $A$ at time $t + t_1 + t_2$ is
%%
%%\begin{eqnarray}
%%	\omega_2 & = & \omega_1 \left( \left( 1 - \frac{2(M - \delta M)}{r_A(t+t_1 +t_2)} \right)^{-\frac{1}{2}} + \frac{\delta M}{r_0} \left( 1 - \frac{2(M - \delta M)}{r_A(t + t_1 + t_2)} \right)^{-1} \right)
%%\end{eqnarray}
%%
%%we can substitute in for $\omega_1$ to find $\omega_2$ in terms of $\omega_\infty$ so that we can compare $\omega_2$ with $\omega_0$. 
%%
%%\begin{eqnarray}
%%	\omega_2 & = & \omega_\infty \left( \left( 1 - \frac{2(M - \delta M)}{r_B(t+t_1)} \right)^{-\frac{1}{2}} + \frac{\delta M}{r_0} \left( 1 - \frac{2(M - \delta M)}{r_B(t+t_1)} \right)^{-1} \right) \left( \left( 1 - \frac{2(M - \delta M)}{r_A(t+t_1 +t_2)} \right)^{-\frac{1}{2}} + \frac{\delta M}{r_0} \left( 1 - \frac{2(M - \delta M)}{r_A(t + t_1 + t_2)} \right)^{-1} \right)	\nonumber	\\
%%	& = & \omega_\infty \left( \left( 1 - 2(M - \delta M) \left( \frac{1}{r_B(t + t_1)} + \frac{1}{r_A(t + t_1 + t_2)} \right) + \frac{4(M - \delta M)^2}{r_B(t + t_1) r_A(t + t_1 +t_2)} \right)^{-\frac{1}{2}}  \right. \nonumber	\\
%%	& + &  \left. \frac{\delta M}{r_0} \left( 1 + \frac{M - \delta M}{r_B(t + t_1)} \right) \left( 1  +\frac{2(M - \delta M)}{r_A(t + t_1 + t_2)} \right) + \frac{\delta M}{r_0 + d} \left( 1 + \frac{2(M - \delta M)}{r_B(t + t_1)} \right) \left( 1 + \frac{M - \delta M}{r_A(t + t_1 + t_2)} \right) + \frac{\delta M^2}{r_0^2} \right)	\nonumber	 \\
%%	& = & \omega_\infty \left( 1 + (M  - \delta M) \left( \frac{1}{r_B(t+t_1)} + \frac{1}{r_A(t + t_1 + t_2} \right) - \frac{2(M - \delta M)^2}{r_B(t + t_1) r_A(t + t_1 + t_2)} \right.\nonumber	\\
%%	& + & \left. \frac{3}{4} (M - \delta M) \left( \frac{1}{r_B(t + t_1)} + \frac{1}{r_A(t + t_1 + t_2)} \right) + \frac{\delta M}{r_0} + \frac{\delta M ( M - \delta M)}{r_0 r_B(t + t_1)} + 2 \frac{\delta M (M - \delta M)}{r_0 r_A(t + t_1 + t_2)} + \frac{\delta M}{r_0} + \frac{2\delta M (M - \delta M)}{r_0 r_B(t + t_1)} \right. \nonumber		\\
%%	& + & \left. \frac{\delta M(M - \delta M)}{r_0 r_A(t + t_1 + t_2)} - \frac{\delta M d}{r_0^2} + \frac{\delta M^2}{r_0^2} \right)
%%\end{eqnarray}
%%
%\subsection{Proper time and distance relation}
%
%Lets compute the proper time it would take a free falling observer
%
%\begin{eqnarray}
%	t_1 & = & r_B(t + t_1) - r_A(t)	\nonumber	\\
%	& = & d - \frac{1}{2} \frac{M}{(r_0 + d)^2} (t + t_1)^2 + \frac{1}{2} \frac{M}{r_0^2} t^2	\nonumber	\\
%	& = & d - \frac{Mdt}{r_0^2} - \frac{1}{2} \frac{Md^2}{r_0^2}
%\end{eqnarray}
%
%\begin{eqnarray}
%	t_2 & = & r_B(t + t_1) - r_A(t + t_1 + t_2)	\nonumber	\\
%	& = & d + \frac{Mdt}{r_0^2} + \frac{3}{2} \frac{Md^2}{r_0^2}
%\end{eqnarray}
%
%Adding the two times gives
%
%\begin{equation}
%	t_1 + t_2 = 2 d + \frac{Md^2}{r_0^2} = 2d \left( 1+ \frac{1}{2} \frac{Md}{r_0^2} \right)
%\end{equation}
%
%The proper time is
%
%\begin{eqnarray}
%	\tau & = & - \left( 1 - \frac{2M}{r_0} \right)^{-\frac{1}{2}} (t_1 + t_2)	\nonumber	\\
%	& = & -2d \left( 1 + \frac{M}{r_0} + \frac{3}{2} \frac{M^2}{r_0^2} + \frac{1}{2} \frac{Md}{r_0^2} \right)
%\end{eqnarray}
%
%the change in distance is just
%
%\begin{eqnarray}
%	r_B(t + t_1) - r_A(t + t_1) = d - \frac{1}{2} \frac{M}{r_0^2}(t + t_1)^2 + \frac{1}{2} \frac{M}{r_0^2} (t + t_1)^2 = d
%\end{eqnarray}
%
%\subsection{Distance between $A$ and $C$}
%
%Lets start with the trajectory of particle $C$, perpendicular to the radial direction, which is a distance $d$ from $A$ 
%
%\begin{equation}
%	r_C(t)_{\bot} = d - \frac{1}{2} \frac{\delta M}{\sqrt{r_0^2 + d^2}} t \sin \theta - \frac{1}{2} \frac{M - \delta M}{r_0^2} t^2 \sin \theta = d - \frac{1}{2} \frac{\delta M}{r_0^2} d t - \frac{1}{2} \frac{(M - \delta M)d}{r_0^3} t^2 
%\end{equation}
%
%we know the acceleration term will be $\frac{1}{2} \frac{M}{r_0^2} t^2 \sin \theta$ therefore it will be suppressed by a further factor of $r_0$. The perpendicular trajectory for $A$
%
%\begin{equation}
%	r_A(t)_{\bot} = 0 
%\end{equation}
%
%We compute the times as we did before
%
%\begin{equation}
%	t_1  =  r_C(t+ t_1)_{\bot} - r_A(t)_{\bot} = r_C(t + t_1)_\bot = t_2
%\end{equation}
%
%\begin{eqnarray}
%	t_2 & = & r_C(t+ t_1)_{\bot} - r_A(t+t_1+t_2)_{\bot} = r_C(t + t_1)_\bot	\nonumber	\\
%	& = &  d - \frac{\delta M}{r_0^2} d (t+t_1) - \frac{1}{2} \frac{(M - \delta M)d}{r_0^3} (t+t_1)^2 
%\end{eqnarray}
%
%The proper time is 
%
%\begin{eqnarray}
%	\tau & = & \left( 1 - \frac{2(M - \delta M)}{r_0} \right)^{ - \frac{1}{2}} ( t_1 + t_2 )	\nonumber	\\
%	& = & 2 \left( 1 - \frac{2(M - \delta M)}{r_0} \right)^{ - \frac{1}{2}} r_C(t+t_1)	\nonumber	\\
%	& = & 2 \left(d + \frac{M - \delta M}{r_0} d + \frac{3}{2} \frac{(M - \delta M)^2 d}{r_0^2} - \frac{1}{2} \frac{M - \delta M}{r_0^3} d (t + t_1)^2 - \frac{\delta M d}{r_0^2} (t+ t_1) \right)	\nonumber	\\
%	& = & 2 d \left(1 + \frac{M - \delta M}{r_0}  + \frac{3}{2} \frac{(M - \delta M)^2 }{r_0^2} - \frac{1}{2} \frac{M - \delta M}{r_0^3} (t + d)^2 - \frac{\delta M }{r_0^2} (t+ d) \right)
%\end{eqnarray}
%
%For just background time trajectory there would be no velocity so just acceleration would give
%
%\begin{eqnarray}
%	\bar{\tau} & = & - 2 \left( 1 - \frac{2M}{r_0} \right)^{-\frac{1}{2}}r_C(t+t_1)	\nonumber	\\
%	& = & 2 \left( 1 - \frac{2M}{r_0} \right)^{-\frac{1}{2}} \left( d  -\frac{1}{2} \frac{M}{r_0^3} d (t + d)^2 \right)	\nonumber	\\
%	& = & -2d \left( 1 + \frac{M}{r_0} + \frac{3}{2} \frac{M^2}{r_0^2} - \frac{1}{2} \frac{M}{r_0^3} (t + d)^2 \right)
%\end{eqnarray}
%
%The difference between these proper times is
%
%\begin{eqnarray}
%	\tau - \bar{\tau} & = & -2d \left( - \frac{\delta M}{r_0} + \frac{3}{2} \frac{\delta M^2 -2 M \delta M}{r_0^2} + \frac{1}{2} \frac{\delta M}{r_0^3} (t + d)^2 - \frac{\delta M}{r_0^2} (t+d) \right)
%\end{eqnarray}
%
%\subsection{Geodesics in Schwarschild}
%
%The Lagrangian for a material particle is moving in radial direction. 
%
%\begin{equation}
%	L = 1 = - \left( 1 - \frac{2M}{r} \right) \dot{t}^2 + \left( 1 - \frac{2M}{r} \right)^{-1} \dot{r}^2
%\end{equation}
%
%The EOM for $t$ is 
%
%\begin{equation}
%	\frac{dL}{dt} = \frac{\partial}{\partial \tau} \frac{dL}{d \dot{t}} 
%\end{equation}
%
%\begin{equation}
%	\frac{\partial}{\partial \tau} \left( 2 \left( 1 - \frac{2M}{r} \right) \dot{t} \right) = 0 \Rightarrow \left( 1 - \frac{2M}{r} \right) \dot{t} \equiv E
%\end{equation}
%
%where $E$ is the conserved quantity w.r.t $t$. The metric/Lagrangian now tells us
%
%\begin{equation}
%	1 = - \left( 1 - \frac{2M}{r} \right)^{-1} E^2 + \left( 1 - \frac{2M}{r} \right)^{-1} \dot{r}^2 = - \left( 1 - \frac{2M}{r} \right)^{-1} \left( E^2 - \dot{r}^2 \right)
%\end{equation}
%
%The EOM for $r$
%
%\begin{equation}
%	\frac{dL}{dr} = - \left( \frac{2M}{r^2} \right) \left( 1 - \frac{2M}{r} \right)^{-2} \dot{r}^2
%\end{equation}
%
%\begin{equation}
%	\frac{\partial}{\partial \tau} \frac{dL}{d \dot{r}} = \frac{\partial}{\partial \tau} \left( 2 \dot{r} \left( 1- \frac{2M}{r} \right) \right)  = \frac{2M}{r^2} \dot{r}^2 + 2 \ddot{r} \left( 1 - \frac{2M}{r} \right) 
%\end{equation}
%
%Combining these terms
%
%\begin{equation}
%	\frac{2M \dot{r}^2}{r^2} + 2 \ddot{r} \left( 1 - \frac{2M}{r} \right) = - \left( \frac{2M}{r^2} \right) \left( 1 - \frac{2M}{r} \right)^{-2} \dot{r}^2
%\end{equation}
%
%A better way to write the geodesic equation:
%
%\begin{eqnarray}
%	\frac{\partial^2 r}{\partial \tau}  + \Gamma^r_{\mu \nu} \frac{\partial x^\mu}{\partial \tau} \frac{\partial x^\nu}{\partial \tau} =  \frac{\partial^2 r}{\partial \tau} + \Gamma^r_{tt} \left( \frac{dt}{d \tau} \right)^2 - \Gamma^r_{rr} \left( \frac{\partial r}{\partial \tau} \right)^2 = 0
%\end{eqnarray}
%
%\begin{eqnarray}
%	& & \ddot{r} + \frac{M}{r^3}(r-2M) \dot{t}^2 - \frac{M\dot{r}^2}{r(r-2M)} = 0 	\nonumber	\\
%	& & \ddot{r} + \frac{M}{r^3} \frac{(r-2M)E^2}{\left( 1 - \frac{2M}{r} \right)^2} - \frac{M \dot{r}^2}{r(r-2M)}= 0  	\nonumber	\\
%	& & \ddot{r} + \frac{M}{r^\frac{5}{2}} \frac{E^2}{r - 2M} - \frac{M}{r(r-2M)} \left( E^2 + \frac{r - 2M}{r} \right) = 0	\nonumber	\\
%	& & \ddot{r} + \frac{M}{r- 2M} \left( \frac{E^2}{r} \left( \frac{1}{r^\frac{3}{2}} - 1\right) - \frac{r- 2M}{r^2} \right) = 0
%\end{eqnarray}
%
%From Hartle Eq 9.39 we get
%
%\begin{equation}
%	\frac{dt}{dr} = - \left( \frac{2M}{r} \right)^{-\frac{1}{2}} \left( 1  -\frac{2M}{r} \right)^{-1}
%\end{equation}
%
%expanding to leading order in $M/r$ 
%
%\begin{equation}
%	\int dt = - \int dr \ \left( \frac{r}{2M} \right)^\frac{1}{2} + \left( \frac{2M}{r} \right)^\frac{1}{2}
%\end{equation}
%
%integrating
%
%\begin{equation}
%	t = -\frac{2}{3 \sqrt{2M}} \left( r^\frac{3}{2} + \frac{1}{2} \sqrt{r} \right)
%\end{equation}
%
%\begin{equation}
%	\frac{3}{2} Mt^2 = r(r + 1/2)^2
%\end{equation}
%
%for very large $r$ we can ignore the factor of $1/2$ to get
%
%\begin{equation}
%	r(t) = \frac{3}{2} \frac{Mt^2}{r^2}
%\end{equation} 
%
%\subsection{Calculating the velocity from IJC}
%
%\begin{equation}
%	g_{\mu \nu} \zeta^\mu_{i}  \Sigma^\nu_{i} = \hat{g}_{\mu \nu} \zeta^\mu_{f}  \Sigma^\nu_{f} 
%\end{equation}
%
%\begin{equation}
%	g_{00} \zeta^0_{i}  \Sigma^0_{i} + g_{11} \zeta^1_{i}  \Sigma^1_{i} = \hat{g}_{00} \zeta^0_{f}  \Sigma^0_{f}  + \hat{g}_{11} \zeta^1_{f}  \Sigma^1_{f}
%\end{equation}
%
%\begin{equation}
%	- \left( 1 - \frac{2M}{r} \right) \left( 1 - \frac{2M}{r} \right)^{-1} \left( 1 - \frac{2M}{r} \right)^{-\frac{1}{2}} = - \left( 1 - \frac{2(M - \delta M)}{r} \right) \left( 1 - \frac{2(M - \delta M)}{r} \right)^{-1} \left( 1 - \frac{2(M - \delta M)}{r} \right)^{-\frac{1}{2}} + v \left( 1 - \frac{2(M - \delta M)}{r_0} \right)^{-1}
%\end{equation}
%
%\begin{equation}
%	- \left( 1 + \frac{M}{r} + \frac{3}{2} \frac{M^2}{r^2} \right) = - \left( 1 - \frac{2(M - \delta M)}{r} \right)^{-\frac{1}{2}} + v \left( 1 - \frac{2(M - \delta M)}{r_0} \right)^{-1}
%\end{equation}
%
%\begin{equation}
%	- \frac{\delta M}{r_0} - \frac{3 \delta M M }{r_0^2}+ \frac{3}{2} \frac{\delta M^2}{r_0^2} = v \left( 1 + \frac{2(M - \delta M)}{r_0 }+ 4 \frac{(M - \delta M)^2}{r_0^2} \right)
%\end{equation}
%
%\begin{equation}
%	v = \left(  - \frac{\delta M}{r_0} - \frac{3 \delta M M }{r_0^2} + \frac{3}{2} \frac{\delta M^2}{r_0^2} \right) \left( 1 + \frac{2(M - \delta M)}{r_0} + \frac{4(M - \delta M)^2}{r_0^2} \right)^{-1}
%\end{equation}
%
%\begin{equation}
%	v = - \frac{\delta M}{r_0} - \frac{M \delta M}{r_0^2} - \frac{1}{2} \frac{\delta M^2}{r_0^2}
%\end{equation}
%
%\newpage
%	
%\section{Full calculation change in proper time}
%
%We would expect that the any difference in proper times would appear at $\mathcal{O}(r_E^{-2})$ thus all quantities are expanded to this order in the calculation. 
%
%\subsection{Before Shell Crossing}
%
%We start by defining the metric
%
%\begin{equation}
%ds^2 = -\left(1-\frac{2M_i}{r}\right)dt^2 + \frac{dr^2}{1-\frac{2M_i}{r}}+r^2d\Omega_2^2~.
%\end{equation}
%
%The two end-points of an interferometer arm, $r_1(t), r_2(t)$ are defined as
%
%\begin{eqnarray}
%r_1(t) &=& r_{E}~, \ \ \ r_2(t) = r_{E}+d~.
%\end{eqnarray}
%
%The equation of motion for outgoing photons travelling radially is obtained from the metric by setting $ds=0$,
%
%\begin{eqnarray}
%t-t_0 &=& r-r_0 + 2M_i\ln\frac{r-2M_i}{r_0-2M_i} \\ \nonumber
%&=& d + 2M_i\frac{d}{r_E} -M_i\frac{d}{r_E}
%\left( \frac{d}{r_E}-\frac{4M_i}{r_E} \right)~.
%\label{eq-dt1}
%\end{eqnarray}
%
%In the last step we plug in $r_0=r_E$, $r = r_E+d$. Likewise, an incoming null-ray has an equation of motion
%
%\begin{eqnarray}
%t-t_0 &=& r_0-r + 2M_i\ln\frac{r_0-2M_i}{r-2M_i} \\ \nonumber
%&=& d + 2M_i\frac{d}{r_E} - M_i\frac{d}{r_E}
%\left( \frac{d}{r_E}-\frac{4M_i}{r_E} \right) ~.
%\end{eqnarray}
%
%Here we used $r_0 = r_E+d$, $r = r_E$. Thus the total coordinate time it takes for a light ray to come back at $r_1$ is
%
%\begin{equation}
%\Delta t = 2d + 4M_i\frac{d}{r_E} - 2M_i\frac{d}{r_E}
%\left( \frac{d}{r_E}-\frac{4M_i}{r_E} \right)~.
%\end{equation}
%
%The total proper time traversed by the first end of the interferometer arm in the time it takes the photon to travel the length of the arm and come back is 
%
%\begin{eqnarray}
%\Delta \tau_i &=& \sqrt{1-\frac{2M_i}{r_E}}\Delta t  \\ \nonumber
%&=& 2d \left(1-\frac{M_i}{r_E} - \frac{M_i^2}{2r_E^2}\right) 
%\left[ 1+\frac{2M_i}{r_E} - \frac{M_i}{r_E}\left( \frac{d}{r_E} - \frac{4M_i}{r_E} \right) \right] \\ \nonumber
%&=& 2d
%\left(1 + \frac{M_i}{r_E} + \frac{3}{2}\frac{M_i^2}{r_E^2} - \frac{M_id}{r_E^2}\right)~.
%\end{eqnarray}
%
%\subsection{After Shell Crossing}
%
%The metric after shell crossing comes from the mass $M_f$,
%\begin{equation}
%ds^2 = -\left(1-\frac{2M_f}{r}\right)dt^2 + \frac{dr^2}{1-\frac{2M_f}{r}}+r^2d\Omega_2^2~,
%\end{equation}
%with $(M_i-M_f)=\delta M$ the neutrino shell mass. Assuming that $r_1$ crosses the shell at $t=0$, we have\footnote{In theory there is also the acceleration of the interferometer however that will be a higher order affect thus we ignore it.}
%\begin{eqnarray}
%r_1(t) &=& r_E - \frac{\delta M}{r_E} \left( 1 - \frac{2M_f}{r_E} \right)^\frac{1}{2}t~,
%\end{eqnarray}
%where $\left( 1 - \frac{2M_f}{r_E} \right)^\frac{1}{2}$ is the conversion factor between proper velocity and coordinate velocity, since this will be a higher order effect and will always accompany $\frac{\delta M}{r_E}$ this term will only affect terms of $\mathcal{O}(r_E^{-2})$. On the other hand, $r_2$ crosses the shell later. We can only derive that by first knowing the outgoing null-ray trajectory,
%\begin{eqnarray}
%t-t_0 = r-r_0 + 2M_f\ln\frac{r-2M_f}{r_0-2M_f}~.
%\label{eq-null1out}
%\end{eqnarray}
%Plugging $t_0=0$, $r_0 = r_1(0) = r_E$, $r = r_2 = r_E+d$, we obtain the crossing time,
%\begin{eqnarray}
%\tilde{t} &=& d + 2M_f\ln \frac{r_E+d-2M_f}{r_E-2M_f} \\ \nonumber
%&=& d + 2M_f\frac{d}{r_E} - M_f\frac{d}{r_E}\left( \frac{d}{r_E} - \frac{4M_f}{r_E} \right)~.
%\end{eqnarray}
%Using this we can calculate $r_2(t)$ 
%\begin{eqnarray}
%r_2(t) &=& r_E + d - \frac{\delta M}{r_E+d} \left( 1 - \frac{2M_f}{r_E+d} \right)^\frac{1}{2}(t-\tilde{t}) \\ \nonumber
%&=& r_E + d - \frac{\delta M}{r_E}(t-d) + \frac{\delta M}{r_E^2} ((M_f + d)t + (M_f - d)d)
%\end{eqnarray}
%Now, for a light ray that starts from $r_1$ at a later time $t_0$, we solve where $r_2$ crosses the null ray by substituting $r_0 = r_1(t_0) = r_E - \frac{\delta M}{r_E}t_0~$ and $r =r_2(t) = r_E + d - \frac{\delta M}{r_E}(t-d) + \frac{\delta M}{r_E}\frac{d}{r_E}(t-d+2M_f)$ into Eq (\ref{eq-null1out}). A further approximation we may use is $t_0\gg d\approx(t-t_0)$. Thus when a term is already second order, we ignore the difference between $t$ and $t_0$. We can then start to solve the crossing time at $r_2$, while carefully keeping all the expansions up to the second order.
%
%\begin{eqnarray}
%& & \Delta t_{out} \equiv  t - t_0 = d - \frac{\delta M}{r_E}(t - t_0 - d) + \frac{\delta M M_f}{r_E^2} (t - t_0 -d)	\nonumber	\\
%&  & + \frac{\delta M}{r_E}\frac{d}{r_E}(t-d+2M_f) \label{eq-SolveCross} \nonumber	\\
%& & + 2M_f\ln\left(1+\frac{d}{r_E}-\frac{2M_f}{r_E} -\frac{\delta M(t-d)}{r_E^2} \right) 
%\nonumber \\ \nonumber
%& & - 2M_f\ln\left(1-\frac{2M_f}{r_E} -\frac{\delta Mt_0}{r_E^2} \right)
%\\ \nonumber
%&=& d - \frac{\delta M}{r_E}(t - t_0 - d) + \frac{M_f \delta M d}{r_E^2} + \frac{\delta M t_0}{r_E^2}(d-M_f) \\ \nonumber
%& & + 2M_f \left[ \frac{d}{r_E}-\frac{2M_f}{r_E} -\frac{\delta M(t-d)}{r_E^2}  - \frac{1}{2} \left( \frac{d}{r_E}-\frac{2M_f}{r_E} \right)^2 \right] \\ \nonumber
%& & - 2M_f \left( -\frac{2M_f}{r_E} -\frac{\delta Mt_0}{r_E^2} - \frac{2M_f^2}{r_E^2} \right) + \frac{\delta M M_f t}{r_E^2} ~.
%\end{eqnarray}
%This leads to
%\begin{eqnarray}
%& & \Delta t_{out} \left(1 + \frac{\delta M}{r_E} - \frac{\delta M M_f}{r_E^2}\right) = 
%d + 2M_f\frac{d}{r_E}  \nonumber	\\
%& & - M_f\frac{d}{r_E} \left( \frac{d}{r_E}-\frac{4M_f}{r_E} \right) 	\nonumber	\\
%& & + \frac{\delta M}{r_E}d + \frac{\delta Md}{r_E^2} (t_0+3M_f)~.
%\end{eqnarray}
%
%Compare this to Eq.~(\ref{eq-dt1}), there are only three differences, and two of them have clear physical meanings. The last term on the right-hand-side is the total extra distance between the two end-points, since their velocities are slightly different. The extra factor on the left-hand-side comes from the fact that both end-points are falling toward the center with roughly identical velocity, so it takes a light ray less time to go from inside to outside. On the other hand, the outside endpoint did not pick up this velocity at $t=0$, but at $t\sim d$. That leads to the second last term which will cancel the previous factor at the leading order. The last two things will of course be reversed in the reflection, while the distance increase stays the same. An incoming null-ray has the trajectory
%
%\begin{eqnarray}
%t_0 - t = r-r_0 + 2M_f\ln\frac{r-2M_f}{r_0-2M_f}~.
%\end{eqnarray}
%
%This allows us to calculate the radial coordinates of the end points of the interferometer arms 
%\begin{eqnarray}
%r_0 &=& r_2(t_0) = r_E +d - \frac{\delta M}{r_E}(t_0-d) \nonumber	\\
%& + & \frac{\delta M}{r_E^2} (d(M_f -d) + t_0(M_f+d)) ~, \nonumber \\
%r &=& r_1(t) = r_E - \frac{\delta M}{r_E}t - \frac{M_f \delta M}{r_E^2} ~.
%\end{eqnarray}
%and the time it takes for an incoming photon to cross the arm, 
%\begin{eqnarray}
%\Delta t_{in} & \equiv & t-t_0 = d + \frac{\delta M}{r_E}(t-t_0+d)	\nonumber	\\
%& +& \frac{\delta M}{r_E^2} \left( - M_f t + t_0 (M_f + d) + d(M_f -d) \right)
%\nonumber  \\
%& & +2M_f\ln\left( 1+\frac{d}{r_E} - \frac{2M_f}{r_E} - \frac{\delta M}{r_E^2}(t_0-d) \right)
%\nonumber \\ \nonumber
%& & -2M_f\ln\left( 1 - \frac{2M_f}{r_E} - \frac{\delta M}{r_E^2}t \right)
%\end{eqnarray}
%Compare this with Eq.~(\ref{eq-SolveCross}), the log terms are identical up to the higher order difference in $t$ and $t_0$ which we can ignore.
%\begin{eqnarray}
%& & \Delta t_{in}\left(1-\frac{\delta M}{r_E} + \frac{M_f \delta M}{r_E^2} \right) =
%d + 2M_f\frac{d}{r_E} 
%\nonumber	\\
%& & - M_f\frac{d}{r_E} \left( \frac{d}{r_E}-\frac{4M_f}{r_E} \right) 
%+ \frac{\delta M}{r_E}d \nonumber	\\
%& +&  \frac{\delta Md}{r_E^2} (t_0-d+2M_f)~.
%\end{eqnarray}
%Combining them, the total duration in coordinate time is
%\begin{eqnarray}
%& & \Delta t = \Delta t_{out} + \Delta t_{in} \nonumber \\
%&=& d \left(1 - \frac{\delta M}{r_E} + \frac{\delta M^2}{r_E^2} \right) 	\nonumber	\\
%& \times & \left[ 1 + \frac{2M_f}{r_E} + \frac{\delta M}{r_E} -\frac{M_f}{r_E}\left( \frac{d}{r_E}-\frac{4M_f}{r_E} \right) + \frac{\delta M (t_0+2M_f)}{r_E^2} \right] \nonumber  \\
%&+& d \left(1 + \frac{\delta M}{r_E} + \frac{\delta M^2}{r_E^2} \right) \nonumber	\\
%& \times & \left[ 1 + \frac{2M_f}{r_E} + \frac{\delta M}{r_E} -\frac{M_f}{r_E}\left( \frac{d}{r_E}-\frac{4M_f}{r_E} \right)  + \frac{\delta M (t_0-d+2M_f)}{r_E^2} \right]  \nonumber	\\
%&=& 2d [ 1 + \frac{2M_f}{r_E} + \frac{\delta M}{r_E} -\frac{M_f}{r_E}\left( \frac{d}{r_E}-\frac{4M_f}{r_E} \right) \nonumber	\\
%&+ & \frac{\delta M (t_0+2M_f-d/2)}{r_E^2} +\frac{\delta M^2}{r_E^2} ]~.
%\end{eqnarray}
%
%Next we convert this coordinate time into proper time, 
%
%\begin{eqnarray}
%\Delta \tau_f &=& \int_{t_0}^{t_0+\Delta t}
% \sqrt{\left(1-\frac{2M_f}{r_1}\right)dt^2 - \left(1-\frac{2M_f}{r_1}\right)^{-1}dr^2} \nonumber	
% \\ 
%&=& \int_{t_0}^{t_0+\Delta t} 
%\sqrt{\left(1 - \frac{2M_f}{r_E - \frac{\delta M}{r_E}t}\right) - \frac{\delta M^2}{r_E^2} }
%~dt  \nonumber	\\
%&=& \left(1 - \frac{M_f}{r_E} - \frac{\delta M^2}{2r_E^2} - \frac{M_f^2}{2 r_E^2} \right)
%\Delta t \nonumber \\
%&=& 2d ( 1 + \frac{M_f}{r_E} + \frac{\delta M}{r_E} - \frac{\delta M M_f}{r_E^2} + \frac{M_fd}{r_E^2}+ \frac{3}{2} \frac{M_f^2}{r_E^2}	\nonumber	\\
%& + & \frac{\delta M (t_0+2M_f-d/2)}{r_E^2} + \frac{\delta M^2}{2r_E^2} ). \label{123}
%\end{eqnarray}
%
%Finally, we arrive at the difference in proper time, $\Delta \tau = \Delta \tau_i - \Delta \tau_f$, 
%
%\begin{equation}
%	\Delta \tau=  \frac{2d}{r_E^2} \left( \frac{3}{2} (M_i^2 - M_f^2) - \frac{d \delta M}{2} - 3 M_f \delta M - \frac{\delta M^2}{2} - t \delta M  \right).
%\end{equation}
%
%\newpage

\newpage

\section{Relating proper time and proper distance before shell crossing}

%\subsection{Crossing time}
%
%Assuming that $t = 0$ when the shell crosses $r_A$. We assume $r_A(t =0) = r_0$ and $r_B(t=0) = r_0 +d $. Using the metric for a null shell we can calculate the coordinate time, $t_c$, at which the shell crosses point $B$. 
%
%\begin{equation}
%	t_c = \left( 1 + \frac{2(M - \delta M)}{r_0} \right) d
%\end{equation}
%
%To leading order $t_c = d$ as expected. It will turn out that for our calculations we will require the leading order correction to this.  

\subsection{Proper time for photon to go between $A$ and $B$ before shell crossing}

From the metric, we can set $ds = 0$, for the photon to get

\begin{equation}
	\int^{t + t_1}_{t} dt = \int^{r_B(t + t_1)}_{r_A(t)} \frac{dr}{\left( 1 - \frac{2M}{r} \right)}	
\end{equation}

Integrating this gives

\begin{equation}
	t_1 = r_B(t + t_1) - r_A(t) + 2M \ln \left( \frac{r_B(t + t_1) -2M}{r_A(t) - 2M} \right)	\label{B22}
\end{equation}

Now we can expand the log using $\ln(x) \approx (x-1) - \frac{1}{2} (x-1)^2 + ...$. For us $x = \frac{r_B(t + t_1) - 2M}{r_A(t) - 2M} $ Thus $x-1= \frac{r_B(t + t_1) - r_A(t)}{r_A(t) - 2M} = \frac{r_B(t + t_1) - r_A(t) }{r_A(t)} \left( 1 - \frac{2M}{r_A(t)} \right)^{-1}$ and we get

\begin{equation}
	2M \ln \left( \frac{r_B(t + t_1) - 2M}{r_A(t) - 2M} \right) =  2M \left( \frac{r_B(t +t_1) - r_A(t)}{r_A(t)} \left( 1 + \frac{2M}{r_A(t)} \right) - \frac{1}{2} \left( \frac{r_B(t + t_1) - r_A(t)}{r_A(t)} \right)^2 \left( 1 -\frac{2M}{r_A(t)} \right)^{-2} \right)
\end{equation}

Since we only want to keep terms that are $\mathcal{O}(r_0^{-2})$ we can replace $\frac{M}{r_A(t)}$ with $\frac{M}{r_0}$. Thus Eq (\ref{B22}) becomes

\begin{eqnarray}
	t_1 & =  & (r_B(t + t_1) - r_A(t)) \left( 1 + \frac{2M}{r_0} \left( 1 + \frac{2M}{r_0} - \frac{1}{2} \frac{r_B(t + t_1) - r_A(t)}{r_0} \right) \right)	\nonumber	\\
	& = & (r_B(t + t_1) - r_A(t)) \left( 1 + \frac{2M}{r_0} + \frac{4M^2 - M (r_B(t + t_1) - r_A(t))}{r_0^2} \right)
\end{eqnarray}


to leading order $r_B(t + t_1) - r_A(t) = d$ 

\begin{equation}
	t_1 = (r_B(t + t_1) - r_A(t)) \left( 1 + \frac{2M}{r_0} + \frac{4M^2 - M d}{r_0^2} \right)
\end{equation}

\begin{equation}
	r_B(t + t_1) - r_A(t) = d - \frac{1}{2} \frac{M}{r_0^2}(t + d)^2 + \frac{1}{2} \frac{Mt^2}{r_0^2} = d \left( 1 - \frac{1}{2} \frac{Md}{r_0^2} - \frac{Mt}{r_0^2} \right)
\end{equation}

Therefore $t_1$ becomes

\begin{equation}
	t_1 = d \left( 1 + \frac{2M}{r_0} + \frac{4M^2}{r_0^2} - \frac{3}{2} \frac{Md}{r_0^2} - \frac{Mt}{r_0^2} \right)
\end{equation}

Now we can compute $t_2$.

\begin{equation}
	t_2 = (r_B(t + t_1) - r_A(t+t_1 +t_2)) \left( 1 + \frac{2M}{r_0} + \frac{4M^2 - M (r_B(t + t_1) - r_A(t+t_1+t_2))}{r_0^2} \right)
\end{equation}


\begin{eqnarray}
	r_B(t + t_1) - r_A(t + t_1 + t_2) & = & d - \frac{1}{2} \frac{M}{r_0^2} (t + d)^2 + \frac{1}{2} \frac{M}{r_0^2} (t + 2d)^2	\nonumber	\\
	& = & d \left( 1 + \frac{Mt}{r_0^2} + \frac{3}{2} \frac{Md}{r_0^2} \right) 
\end{eqnarray}

Thus $t_2$ becomes

\begin{eqnarray}
	t_2 & = & d \left( 1 + \frac{Mt}{r_0^2} + \frac{3}{2} \frac{Md}{r_0^2} \right) \left( 1 + \frac{2M}{r_0} + \frac{4M^2 - Md}{r_0^2} \right)\nonumber	\\
	& = & d \left( 1 + \frac{2M}{r_0} + \frac{4M^2}{r_0^2} + \frac{1}{2} \frac{Md}{r_0^2} + \frac{Mt}{r_0^2} \right)
\end{eqnarray}

Summing $t_1$ and $t_2$

\begin{equation}
	t_1 +t_2 = 2d \left(1 + \frac{2M}{r_0} + \frac{4M^2}{r_0^2} - \frac{1}{2} \frac{Md}{r_0^2}\right)
\end{equation}

The proper time measured is

\begin{eqnarray}
	\tau & = &  \left( 1 - \frac{2M}{r} \right)^\frac{1}{2} (t_1 + t_2) 	\nonumber	\\
	& = & 2d \left( 1 - \frac{M}{r} - \frac{1}{2} \frac{M^2}{r_0^2} \right) \left( 1 + \frac{2M}{r_0} + \frac{4M^2}{r_0^2} - \frac{1}{2} \frac{Md}{r_0^2} \right)		\nonumber	\\
	&  = & 2d \left( 1  + \frac{M}{r_0} + \frac{3}{2} \frac{M^2}{r_0^2} - \frac{1}{2} \frac{Md}{r_0^2} \right)
\end{eqnarray}

%
%\subsection{Proper distance between $A$ and $B$ before shell crossing}
%
%In this section we calculate the proper distance $\bar{l}_{AB}$ between the two points $A$ and $B$ at time $\bar{t}$. This calculation is done without any shell crossing, so everything is in the metric of $M$. 
%
%\begin{equation}
%	\bar{l}_{AB} = \int^{r_B(\bar{t})}_{r_A(\bar{t})} dr \ \bar{g}_{rr}^\frac{1}{2}  = \int^{r_B(\bar{t})}_{r_A(\bar{t})} dr \left( 1 + \frac{M}{r} + \frac{3}{2} \frac{M^2}{r^2}  + \mathcal{O}((M/r)^3) \right).
%\end{equation}
%
%In all the calculations we will only need terms of $\mathcal{O}(r_0^{-2})$. Thus we can use $r_B(\bar{t}) = r_0 + d$ and $r_A(\bar{t})  = r_0$. 
%
%\begin{eqnarray}
%	  \int^{r_B(\bar{t})}_{r_A(\bar{t})}  dr \ \bar{g}_{rr}^\frac{1}{2} & = & d + M \ln \left( \frac{r_0 + d}{r_0} \right) - \frac{3}{2} M^2 \left( \frac{1}{r_0 + d} - \frac{1}{r_0} \right)	\nonumber	\\
%	  & = & d + M \left( \frac{d}{r_0} - \frac{1}{2} \frac{d^2}{r_0^2} \right) + \frac{3}{2} \frac{M^2 d}{r_0^2} \nonumber	\\
%	  & = & d \left( 1 + \frac{M}{r_0} - \frac{1}{2} \frac{Md}{r_0^2} + \frac{3}{2} \frac{M^2}{r_0^2} \right)
%\end{eqnarray}
%
%We note that $2 \bar{l}_{AB} = - \tau_{AB}$. 

\newpage

\section{Proper length after shell crossing}

In this section we calculate the proper length between $A$ and $B$ after shell crossing. Integrating Eq (\ref{proplen}) with the trajectories for $r_B(t)$ and $r_A(t)$ gives 

\begin{equation}
	l_{AB} = r_B(t) - r_A(t) + (M - \delta M) \underbrace{\ln \left( \frac{r_B(t)}{r_A(t)} \right)}_{T1} - \underbrace{\frac{3}{2} (M - \delta M)^2 \left( \frac{1}{r_B(t)} - \frac{1}{r_A(t)} \right)}_{T2}
\end{equation}
Thus we see the only new thing to compute is $r_B(t) - r_A(t)$. Lets expand these terms individually, starting with $T1$

\begin{eqnarray}
	\ln \left( \frac{r_B(t)}{r_A(t)} \right) & = & \ln \left( \frac{(r_B - r_0) + r_0}{(r_A - r_0) + r_0} \right)	\nonumber	\\
	& = & \ln \left( \left( 1 + \frac{r_B - r_0}{r_0} \right) \left( 1 + \frac{r_A - r_0}{r_0} \right)^{-1} \right)	\nonumber	\\
	& = & \ln \left( 1 + \frac{r_B - r_A}{r_0} - \frac{(r_B - r_0)(r_A - r_0)}{r_0^2} - \left( \frac{r_A - r_0}{r_0} \right)^2 \right).
\end{eqnarray}
Note that $r_A - r_0 = \mathcal{O}(r_0^{-1})$ and we are only interested in terms up to $\mathcal{O}(r_0^{-2})$ thus we simplify the above equation to 
\begin{equation}
	\ln \left( \frac{r_B(t)}{r_A(t)} \right) = \ln \left( 1 + \frac{r_B - r_A}{r_0} \right).
\end{equation}
Now we look ahead to $r_B(t) - r_A(t)$ in Eq (\ref{rchange}). We will need to keep terms up to $\mathcal{O}(r_0^{-1})$ in $r_B(t) - r_A(t)$. Thus we expand the log to get

\begin{equation}
	\ln \left( \frac{r_B(t)}{r_A(t)} \right) = \frac{r_B - r_A}{r_0} - \frac{1}{2} \left( \frac{r_B - r_A}{r_0} \right)^2 = \frac{d}{r_0} + \frac{\delta M d}{r_0^2} - \frac{1}{2} \frac{d^2}{r_0^2} 
\end{equation}
Evaluating $T2$ to $\mathcal{O}(r_0^{-2})$ is straightforward

\begin{equation}
	- \frac{3}{2} (M - \delta M)^2 \left( \frac{1}{r_B(t)} - \frac{1}{r_A(t)} \right) = \frac{3}{2} \frac{(M -\delta M)^2d}{r_0^2}	\label{rsecond}
\end{equation}
Putting in the expansion of $T1$ and $T2$ into $l_{AB}$

\begin{equation}
	l_{AB} = r_B(t) - r_A(t) + \frac{(M - \delta M)d}{r_0} + \frac{\delta M (M - \delta M)d}{r_0^2} - \frac{1}{2} \frac{d^2(M- \delta M)}{r_0^2} + \frac{3}{2} \frac{(M - \delta M)^2d }{r_0^2}
\end{equation}

Now we can compute $r_B(t) - r_A(t)$. 
\begin{eqnarray}
	r_B(t) - r_A(t) & = & d - \frac{1}{2} \frac{M}{r_0^2} t_c^2 - \left( \frac{M}{r_0^2} t_c + \frac{\delta M}{r_0} + \frac{\delta M^2}{r_0^2} - \frac{\delta M d}{r_0^2} \right) (t - t_c) - \frac{1}{2} \frac{(M - \delta M)}{r_0^2} (t - t_c)^2 	\nonumber	\\
	& + & \frac{1}{2} \frac{(M - \delta M) t^2}{r_0^2} + \left( \frac{\delta M}{r_0} + \frac{\delta M^2}{r_0^2} \right) t	\nonumber	\\
	& = & d + \frac{\delta M t_c}{r_0} + \frac{\delta M^2 d}{r_0^2} - \frac{1}{2} \frac{\delta M d^2}{r_0^2} 	\nonumber	\\
	& = & d + \frac{\delta M t_c}{r_0} + \frac{\delta M^2 d}{r_0^2} - \frac{1}{2} \frac{\delta M d^2}{r_0^2} 		\label{rchange}
\end{eqnarray}

Finally we get the expression for $l_{AB}$ 
\begin{equation}
	l_{AB} = d + \frac{\delta M t_c}{r_0} + \frac{\delta M^2 d}{r_0^2} - \frac{1}{2} \frac{ \delta M d^2}{r_0^2} + \frac{(M - \delta M) d}{r_0} + \frac{\delta M (M - \delta M)d}{r_0^2} - \frac{1}{2} \frac{d^2(M - \delta M)}{r_0^2} + \frac{3}{2} \frac{(M - \delta M)^2 d}{r_0^2}
\end{equation}

Substituting in for $t_c = d \left( 1+ \frac{2(M - \delta M)}{r_0} \right)$ gives
\begin{eqnarray}
	l_{AB} & = & d + \frac{\delta M d}{r_0} + \frac{2\delta M(M - \delta M)d}{r_0^2} + \frac{\delta M^2}{r_0^2} d - \frac{1}{2} \frac{\delta M d^2}{r_0^2} + \frac{(M - \delta M)d}{r_0} + \frac{\delta M(M - \delta M) d}{r_0^2} - \frac{1}{2} \frac{d^2(M - \delta M)}{r_0^2} 	\nonumber	\\
	& + & \frac{3}{2} \frac{M^2 d}{r_0^2} - \frac{3 M \delta M d}{r_0^2} + \frac{3}{2} \frac{\delta M^2 d}{r_0^2}	\nonumber	\\
	& = & d + \frac{M d}{r_0} - \frac{1}{2} \frac{\delta M^2 d}{r_0^2} + \frac{3}{2} \frac{M^2 d}{r_0^2} -\frac{1}{2} \frac{Md^2}{r_0^2}
\end{eqnarray}


%
%
%
%\newpage
%
%\subsection{old calc}
%
%
%
%
%
%
%The time $t_1$ is again given by
%
%\begin{equation}
%	t_1 = (r_B(t+t_1) - r_A(t)) \left( 1 + \frac{2(M-\delta M)}{r_0} + \frac{4(M-\delta M)^2}{r_0^2} - \frac{d(M-\delta M)}{r_0^2} \right)
%\end{equation}
%
%First lets look at 
%
%\begin{eqnarray}
%	r_B(t + t_1) - r_A(t) & = & d - \frac{1}{2} \frac{M - \delta M}{r_0^2}d^2 - \frac{1}{2} \frac{M - \delta M}{r_0^2}t^2 - \frac{\delta M}{r_0} \left( 1 + \frac{\delta M}{r_0} \right)(t + t_1 -d) + \frac{\delta M d t}{r_0^2} - \frac{Mdt}{r_0^2}  \nonumber	\\
%	& + & \frac{1}{2} \frac{M - \delta M}{r_0^2} t^2 + \frac{\delta M}{r_0} \left( 1 + \frac{\delta M}{r_0} \right) t + \frac{2 \delta M (M - \delta M)d}{r_0^2}	\nonumber	\\
%	& = & d - \frac{1}{2} \frac{M - \delta M}{r_0^2} d^2 - \frac{\delta M}{r_0} t_1 + \frac{\delta M d}{r_0} - \frac{(M - \delta M)dt}{r_0^2} + \frac{2 \delta M (M - \delta M)d}{r_0^2}
%\end{eqnarray}
%
%So $t_1$ becomes
%
%\begin{eqnarray}
%	t_1 & = & \left( d + \frac{2 \delta M (M - \delta M)d}{r_0^2} - \frac{1}{2} \frac{M - \delta M}{r_0^2} d^2 - \frac{\delta M}{r_0} t_1 + \frac{\delta M}{r_0} d - \frac{(M - \delta M) dt}{r_0^2} \right)	\nonumber	\\
%	& \times & \left( 1 + \frac{2(M - \delta M)}{r_0} + \frac{4(M - \delta M)^2}{r_0^2} - \frac{d(M - \delta M)}{r_0^2} \right)	\nonumber	\\
%	& = & d - \frac{\delta M}{r_0} t_1 + \frac{\delta M}{r_0} d + \frac{2\delta M(M - \delta M)d}{r_0^2} - \frac{1}{2} \frac{M - \delta M}{r_0^2} d^2 - \frac{(M - \delta M)dt}{r_0^2} + \frac{2d(M - \delta M)}{r_0} + \frac{4(M - \delta M)^2d}{r_0^2} - \frac{d^2(M - \delta M)}{r_0^2}	\nonumber	\\
%\end{eqnarray}
%	
%\begin{equation}
%	t_1\left( 1 + \frac{\delta M}{r_0} \right) = d \left( 1 +\frac{\delta M}{r_0} + \frac{2\delta M(M - \delta M)}{r_0^2} - \frac{1}{2} \frac{(M - \delta M)d}{r_0^2} - \frac{(M - \delta M)t}{r_0^2} + \frac{2(M - \delta M)}{r_0} + \frac{4(M - \delta M)^2}{r_0^2} - \frac{(M - \delta M)d}{r_0^2} \right)
%\end{equation}
%
%\begin{equation}	
%	t_1 = d \left( 1 + \frac{2(M - \delta M)}{r_0} - \frac{3}{2} \frac{(M - \delta M)d}{r_0^2} - \frac{(M - \delta M)t}{r_0^2} + \frac{4(M - \delta M)^2}{r_0^2} - \frac{2\delta M^2}{r_0^2} \right)
%\end{equation}
%
%Now we can compute $t_2$. 
%
%\begin{equation}
%	t_2 = (r_B(t + t_1) - r_A(t + t_1 + t_2)) \left(1 + \frac{2(M - \delta M)}{r_0} + \frac{4(M - \delta M)^2}{r_0^2} - \frac{d(M - \delta M)}{r_0^2} \right)
%\end{equation}
%
%\begin{eqnarray}
%	r_B(t + t_1) - r_A(t + t_1 + t_2) & = & d - \frac{1}{2} \frac{M - \delta M}{r_0^2}d^2 - \frac{1}{2} \frac{M - \delta M}{r_0^2} t^2 - \frac{\delta M}{r_0} \left( 1 + \frac{\delta M}{r_0} \right) (t + t_1 -d) + \frac{\delta M dt}{r_0^2} - \frac{Mdt}{r_0^2} \nonumber	\\
%	&+ & \frac{1}{2} \frac{M - \delta M}{r_0^2}(t +2d)^2 + \frac{\delta M}{r_0} \left( 1 + \frac{\delta M}{r_0} \right) (t + t_1 + t_2) + \frac{2 \delta M (M - \delta M)d}{r_0^2}	\nonumber	\\
%	& = & d + \frac{3}{2} \frac{M - \delta M}{r_0^2} d^2 + \frac{\delta M d}{r_0} + \frac{\delta M t_2}{r_0} +  \frac{\delta M^2}{r_0^2} d + \frac{(M - \delta M)dt}{r_0^2} + \frac{2 \delta M (M - \delta M)d}{r_0^2}	\nonumber	\\
%\end{eqnarray}
%
%Thus $t_2$ becomes
%
%\begin{eqnarray}
%	t_2 &  =& \left( d + \frac{3}{2} \frac{M - \delta M}{r_0^2}d^2 + \frac{\delta M}{r_0} d + \frac{\delta Mt_2}{r_0} + \frac{\delta M^2}{r_0^2} d + \frac{(M - \delta M)dt}{r_0^2} + \frac{2 \delta M (M - \delta M)d}{r_0^2} \right) \nonumber	\\
%	& \times & \left( 1 + \frac{2(M - \delta M)}{r_0} + \frac{4(M - \delta M)^2}{r_0^2} - \frac{d (M - \delta M)}{r_0^2} \right)	\nonumber	\\
%	& = & d + \frac{3}{2} \frac{(M - \delta M) d^2}{r_0^2} + \frac{\delta M}{r_0} d + \frac{\delta M t_2}{r_0} + \frac{\delta M^2 d}{r_0^2} + \frac{(M - \delta M)dt}{r_0^2} + \frac{2\delta M(M - \delta M)d}{r_0^2} + \frac{2(M - \delta M)d}{r_0} 	\nonumber	\\
%	& & + \frac{2\delta M(M - \delta M)d}{r_0^2} + \frac{2\delta M(M - \delta M)t_2}{r_0^2} + \frac{4(M - \delta M)^2d}{r_0^2} - \frac{d^2(M - \delta M)}{r_0^2}	\nonumber	\\
%\end{eqnarray}
%
%Now we use $t_2 =d$ for the term which is $\mathcal{O}(r_0^{-2}) t_2$. 
%
%\begin{eqnarray}
%	t_2 \left( 1 - \frac{\delta M}{r_0} \right) & = & d \left( 1 + \frac{\delta M}{r_0} + \frac{2(M - \delta M)}{r_0} + \frac{3}{2} \frac{(M - \delta M)d}{r_0^2} + \frac{\delta M^2}{r_0^2} + \frac{(M - \delta M)t}{r_0^2} + \frac{2\delta M(M - \delta M)}{r_0^2}  \right.	\nonumber	\\
%	& + & \left. \frac{2 \delta M(M - \delta M)}{r_0^2} + \frac{2\delta M(M - \delta M)}{r_0^2} + \frac{4(M - \delta M)^2}{r_0^2} - \frac{d(M - \delta M)}{r_0^2} \right)
%\end{eqnarray}
%
%\begin{equation}
%	t_2 = d \left( 1 + \frac{2\delta M}{r_0} + \frac{2(M - \delta M)}{r_0} + \frac{1}{2} \frac{(M - \delta M)d}{r_0^2} + \frac{4 \delta M^2}{r_0^2} + \frac{(M - \delta M)t}{r_0^2} + \frac{8 \delta M(M - \delta M)}{r_0^2} + \frac{4(M - \delta M)^2}{r_0^2} \right)
%\end{equation}
%
%Combining this with $t_1$ gives
%
%\begin{equation}
%	t_1 + t_2 = 2d \left( 1 + \frac{\delta M}{r_0} + \frac{2(M - \delta M)}{r_0} - \frac{1}{2} \frac{(M - \delta M)d}{r_0^2} + \frac{\delta M^2}{r_0^2} + \frac{4\delta M(M - \delta M)}{r_0^2} + \frac{4(M - \delta M)^2}{r_0^2} \right)
%\end{equation}
%
%Now we need to turn this into proper time, which is surprisingly tricky here.
%
%\begin{equation}
%	d \tau^2 = - dt^2 \left( 1 - \frac{2(M - \delta M)}{r_0} \right)^{-1} \left( \left( \frac{dr}{dt} \right)^2 - \left( 1  - \frac{2(M - \delta M)}{r_0} \right)^2 \right)
%\end{equation}
%
%Using the expression for velocity we get
%
%\begin{equation}
%	d \tau = dt \left( 1 - \frac{2(M - \delta M)}{r_0} \right)^{-\frac{1}{2}} \left( - \frac{\delta M^2}{r_0^2} + \left(1 - \frac{2(M - \delta M)}{r_0} \right)^2 \right)^\frac{1}{2}
%\end{equation}
%
%Which we can expand to get
%
%\begin{equation}
%	d \tau = dt \left( 1 - \frac{M - \delta M}{r_0} - \frac{1}{2} \frac{(M - \delta M)^2}{r_0^2} - \frac{1}{2} \frac{\delta M^2}{r_0^2}\right)
%\end{equation}
%
%Thus we get
%
%\begin{eqnarray}
%	\tau & = &-  2d \left( 1 + \frac{\delta M}{r_0} + \frac{2(M - \delta M)}{r_0} - \frac{1}{2} \frac{(M - \delta M)d}{r_0^2} +\frac{\delta M^2}{r_0^2} + \frac{4\delta M(M - \delta M)}{r_0^2} + \frac{4(M - \delta M)^2}{r_0^2} \right)\nonumber	\\
%	& \times & \left( 1 - \frac{(M - \delta M)}{r_0} - \frac{1}{2} \frac{(M - \delta M)^2}{r_0^2} - \frac{1}{2} \frac{\delta M^2}{r_0^2} \right)	\nonumber	\\
%	& = & - 2d \left( 1 + \frac{M}{r_0}  - \frac{1}{2} \frac{(M - \delta M)d}{r_0^2} + \frac{3}{2} \frac{(M - \delta M)^2}{r_0^2} + \frac{1}{2} \frac{\delta M^2}{r_0^2} + \frac{3 \delta M(M - \delta M)}{r_0^2} \right)
%\end{eqnarray}
%	
%Comparing this to the proper time before shell crossing we get 
%
%\begin{equation}
%	\Delta \tau = \tau - \bar{\tau} = \frac{2d \delta M}{r_0^2} (d -  \delta M)
%\end{equation}
%
%\section{Distance between $A$ and $C$}
%
%We follow the same procedure as we did for the distance between $A$ and $B$. First we calculate the time, $t_1$, it takes for a photon starting from $A$ to go to $B$
%
%\begin{equation}
%	t_1 = r_C(t + t_1)_\bot - r_A(t)_\bot = d - \frac{\delta M d(t + d)}{r_0^2}.
%\end{equation}
%
%The time it takes for the photon to go from $C$ to $A$, $t_2$, is also equal to $t_1$. Therefore the proper time is
%
%\begin{equation}
%	\tau_{AC} = 2t_1 \left( 1 - \frac{2(M - \delta M)}{r_0} \right)^\frac{1}{2} = 2d \left( 1 - \frac{\delta M(t +d)}{r_0^2} - \frac{M - \delta M}{r_0} \right)
%\end{equation}
%
%The proper time without any shell crossing would be
%
%\begin{equation}
%	\bar{\tau}_{AC} = - 2d \left(1 - \frac{M}{r_0} \right).
%\end{equation}
%
%The change in proper time is
%
%\begin{equation}
%	\Delta \tau_{AC} \equiv \tau_{AC} - \bar{\tau}_{AC} = - 2d \left( \frac{\delta M(t + d)}{r_0^2} + \frac{\delta M}{r_0} \right)
%\end{equation}
%
%\subsection{Normalization of four-velocity of observer}
%
%The exact, normalized, four vector describing the four velocity of a time-like observer (in our case the arms of the interferometer).\footnote{One check explicitly that it satisfies $u^\mu u^\nu g_{\mu \nu} = -1$ exactlly.}
%
%\begin{equation}
%	u^\mu = \begin{pmatrix} (-1)^{\frac{1}{2}} (v^2 g_{tt}^{-1} +1)^\frac{1}{2} g^{-\frac{1}{2}}_{tt}, v, 0, 0 \end{pmatrix}
%\end{equation}
%
%The other vectors we need are
%
%\begin{eqnarray}
%	& & \bar{u}^\mu = ((-1)^\frac{1}{2} \bar{g}_{tt}^{-\frac{1}{2}}, 0, 0, 0)	\nonumber	\\
%	& & k^\mu = (g_{tt}^{-1}, 1, 0, 0)	\nonumber	\\
%	& & \bar{k}^\mu = (\bar{g}_{tt}^{-1},1,0,0)
%\end{eqnarray}
%	
%	
%Using this we can calculate the junction conditions and the velocity $v$. 
%
%\begin{equation}	
%	k^\mu u^\nu g_{\mu \nu} = \bar{k}^\mu \bar{u}^\nu \bar{g}_{\mu \nu}
%\end{equation}
%
%To begin with we keep the $i$'s and then we just ignore it to get the real part. 
%
%\begin{equation}
%	(-1)^\frac{1}{2} g_{tt}^{-\frac{1}{2}} \sqrt{1-v^2 g_{tt}^{-1}} + vg_{tt}^{-1} = (-1)^\frac{1}{2} \bar{g}_{tt}^{-\frac{1}{2}}
%\end{equation}
%
%Substituting in for the metric and keeping terms to $\mathcal{O}(r_0^{-2})$. Note that we use the fact that to leading order $v = \mathcal{O}(r_0^{-1})$. 
%
%\begin{equation}
%	\left( 1 - \frac{2(M - \delta M)}{r_0} \right)^{-\frac{1}{2}} \left( 1 + v^2 \right)^\frac{1}{2} + v g_{rr} = \left( 1 - \frac{2M}{r} \right)^{-\frac{1}{2}}
%\end{equation}
%
%\begin{equation}
%	\frac{1}{2} v^2 + v \left( 1 + \frac{2(M - \delta M)}{r_0} \right) = \frac{\delta M}{r_0} - \frac{3}{2} \frac{(M - \delta M)^2}{r_0^2} + \frac{3}{2} \frac{M^2}{r_0^2}
%\end{equation}
%
%Now we expand the velocity by orders of suppression by $r_0$. $v = v_1 + v_2$, where $v_1, v_2 = \mathcal{O}(r_0^{-1}), \mathcal{O}(r_0^{-2})$. 
%
%\begin{equation}
%	\frac{1}{2} (v_1 + v_2)^2 + (v_1 + v_2) \left( 1 + \frac{2(M - \delta M)}{r_0} \right) = \frac{\delta M}{r_0} - \frac{3}{2} \frac{(M - \delta M)^2}{r_0^2} + \frac{3}{2} \frac{M^2}{r_0^2} 
%\end{equation}
%
%To first order we get
%
%\begin{equation}
%	v_1 = \frac{\delta M}{r}
%\end{equation}
%
%To second order
%
%\begin{equation}
%	\frac{1}{2} v_1^2 + v_1 \frac{2(M - \delta M)}{r_0} + v_2 = - \frac{3}{2} \frac{(M - \delta M)^2}{r_0^2} + \frac{3}{2} \frac{M^2}{r_0^2} 
%\end{equation}
%
%substituting in $v_1 = \frac{\delta M}{r_0}$ gives
%
%\begin{equation}
%	v_2 = \frac{M \delta M}{r_0^2} 
%\end{equation}
%
%so we get
%
%\begin{equation}
%	v = v_1 + v_2 = \frac{\delta M}{r} \left( 1 + \frac{M}{r} \right)
%\end{equation}
%
%
%
%\newpage
%
%
%\subsection{Normalization of four-velocity of observer}
%
%The exact, normalized, four vector describing the four velocity of a time-like observer (in our case the arms of the interferometer).\footnote{One check explicitly that it satisfies $u^\mu u^\nu g_{\mu \nu} = -1$ exactlly.}
%
%\begin{equation}
%	u^\mu = \begin{pmatrix} (-1)^{\frac{1}{2}} (v^2 g_{tt}^{-1} -1)^\frac{1}{2} g^{-\frac{1}{2}}_{tt}, v, 0, 0 \end{pmatrix}
%\end{equation}
%
%The other vectors we need are
%
%\begin{eqnarray}
%	& & \bar{u}^\mu = ((-1)^\frac{1}{2} \bar{g}_{tt}^{-\frac{1}{2}}, 0, 0, 0)	\nonumber	\\
%	& & k^\mu = (g_{tt}^{-1}, 1, 0, 0)	\nonumber	\\
%	& & \bar{k}^\mu = (\bar{g}_{tt}^{-1},1,0,0)
%\end{eqnarray}
%	
%	
%Using this we can calculate the junction conditions and the velocity $v$. 
%
%\begin{equation}	
%	k^\mu u^\nu g_{\mu \nu} = \bar{k}^\mu \bar{u}^\nu \bar{g}_{\mu \nu}
%\end{equation}
%
%To begin with we keep the $i$'s and then we just ignore it to get the real part. 
%
%\begin{equation}
%	ig_{tt}^{-\frac{1}{2}} \sqrt{1-v^2 g_{tt}^{-1}} - vg_{tt}^{-1} = (-1)^\frac{1}{2} \bar{g}_{tt}^{-\frac{1}{2}}
%\end{equation}
%
%Which has the solution
%
%\begin{equation}
%	v  = - \frac{1}{2} i \bar{g}_{tt}^{-\frac{1}{2}}(g_{tt} - \bar{g}_{tt})
%\end{equation}
%
%which simplifies to
%
%\begin{equation}
%	v = - \frac{\delta M}{r} \left( 1 - \frac{2M}{r} \right)^{-\frac{1}{2}} = - \frac{\delta M}{r} \bar{g}_{tt}^{-\frac{1}{2}}
%\end{equation}
%












	
	
	
	
	
	
	



\bibliography{all_active}

\end{document}